%!TEX root=../../Benutzerhandbuch.tex
\subsection{Schreiben eines Algorithmus}
Um die Software zu bedienen, wird ein in F\# geschriebener Algorithmus ben�tigt. Dieser muss als gls{dll}-Datei zur Laufzeit in die Software eingebunden werden. Eine solche Datei wird mittels Microsoft Visual Studio erzeugt werden, damit die Software funktionieren kann. \\
Folgende Eigenschaften muss die Datei erf�llen:
\begin{itemize}
	\item Name der Methode: startCalculation
	\item �bergabeparameter 1: eine Liste aller historischen Daten 
	\item �bergabeparameter 2: eine Liste der Signale f�r die Entscheidungen (bei �bergabe ist diese Liste leer)
\end{itemize}
Mit dem R�ckgabewert der Methode wird nicht gearbeitet, sondern mit der vom Algorithmus verarbeiteten Signalliste.

\begin{lstlisting}[caption=Dateityp des 1. Parameters]{prices}
System.Collections.Generic.List<System.Tuple
<System.DateTime,decimal,decimal,decimal,decimal>>
\end{lstlisting}
\begin{lstlisting}[caption=Dateityp des 2. Parameters]{signals}
System.Collections.Generic.List<int>
\end{lstlisting}
Au�erdem muss sich der Algorithmus im Namespace  "`Algorithm"' und im Modul "`DecisionCalculator"' befinden.
\subsubsection{Erzeugen eines Algorithmus-Files}
Zuerst muss Microsoft Visual Studio ge�ffnet und ein neues Projekt erzeugt werden.\\
\begin{figure}[!h]
\centering
\includegraphics[width=1\textwidth]{images/newProject.png}
\caption{Erzeugen eines neuen Projekts}
\end{figure} 
\newpage
Erstellen Sie es als F\#-Bibliothek.
\begin{figure}[!h]
\centering
\includegraphics[width=1\textwidth]{images/createProject.png}
\caption{Erstellen Sie es als F\#-Bibliothek}
\end{figure}
H�lt man sich an die im Kapitel "`Schreiben eines Algorithmus"' genannten Richtlinien und implementiert die Methode \inline{startCalculation}, kann das Projekt kompiliert werden und der Algorithmus ist einsatzbereit.
\begin{figure}[!h]
\centering
\includegraphics[width=1\textwidth]{images/makeFs.png}
\caption{Kompilieren des Projekts}
\end{figure}
\subsubsection{Entscheidungen}
Folgende Entscheidungen sind zul�ssig:
\begin{center}
\begin{tabular}{ |  p{1cm} | p{8cm} |}
\hline 
3 & starkes Kaufssignal \\ \hline
2 &   mittleres Kaufssignal \\ \hline
1 &  schwaches Kaufssignal\\ \hline
0 &  alle Best�nde werden gekauft oder verkauft\\ \hline
-1 &  schwaches Verkaufssignal\\ \hline
-2 &  mittleres Verkaufssignal\\ \hline
-3 &  starkes Verkaufssignal\\ \hline
\end{tabular}
\end{center}
