\chapter*{Abstract}

High frequency trading of stocks and other securities often involve multiple transactions per day and depend on both real-time data
and computer-systems to process them.
Methods for analysing price movements and identifying opportune moments to buy or sell exist in abundance and can usually be
implemented in software.\\

This diploma project deals with the development and programming of an algorithmic trading strategy for securities
utilising Technical Analysis and charting techniques. 
To achieve this, diverse indicators such as the Relative Strength Index (RSI), the Average Directional Index (ADX) or the Bollinger Bands, and methods such as the development of trading systems
with Moving Averages (MAs) or regression analysis are described, tested and evaluated. The outcome consists of two dif\-ferent approaches. Both have been gradually improved and are able to trade profitable without human intervention, if adequate parameters are provided.\\

In order to facilitate the evaluation of such algorithms in general and of this algorithm specifically, a software has been developed to
measure the resulting performance from calculations on historical data, which in the following is denoted as Backtesting Software (BTS).
With a graphical user interface, intuitive design and informative charts, the software meets the high requirements and demands of user-friendliness and usability.