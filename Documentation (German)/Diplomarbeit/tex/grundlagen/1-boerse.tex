% !TEX root = ../../Noctua_Diplomarbeit.tex

\subsection{B�rse und Preisbildung}

Eine B�rse ist ein Handelsmarkt, auf dem sich Preise durch Angebot und Nachfrage von Handelspartnern bilden und der Handel nicht direkt zwischen K�ufern und Verk�ufern, sondern �ber berechtigte H�ndler abgewickelt wird. Wichtig ist dabei, dass immer f�r ausreichende Liquidit�t gesorgt werden muss, so dass jederzeit Wertpapiere gekauft und auch verkauft werden k�nnen.\\

Handelsvorg�nge oder Trades resultieren aus einer Erwartungshaltung der Marktteilnehmer.  \cite{boerse-wien-funktion} \cite{boerse-frankfurt-funktion}
Unter der Annahme von steigenden Kursen werden Einheiten gekauft, i.e. es wird "`long gegangen"', werden fallende Kurse erwartet, werden entweder existierende St�ckzahlen verkauft oder es wird sogar ein Leerverkauf get�tigt, i.e. eine Short-Position er�ffnet. (Damit wird der Verkauf von geborgten Wertpapieren bezeichnet, die zu einem sp�teren Zeitpunkt beim Schlie�en der Short-Position erst gekauft werden.) Sinken die Kurse dazwischen, wird daher zu einem h�heren Preis verkauft als sp�ter gekauft wird und es entsteht die Differenz als Gewinn. \cite{gabler-leerverkauf}\\

Wollen mehr Handelsteilnehmer oder \emph{Trader} kaufen als verkaufen, steigt der Preis aufgrund der hohen Nachfrage, man spricht auch von einem Bullenmarkt. \cite{duden-bullenmarkt}
Ist es andersherum so, dass die Verk�ufer �berwiegen und der Preis sinkt, handelt es sich um einen B�renmarkt. \cite{duden-baerenmarkt}\\

F�r jedes Wertpapier wird ein sogenanntes Orderbuch gef�hrt, das die aktuellen Kauf- und Verkaufsauftr�ge beinhaltet. Investoren interessieren sich meist f�r die sogenannte Quote-Zeile, die Informationen zu den g�nstigsten Konditionen sowohl auf K�ufer- als auch auf Verk�uferseite bietet.\\

Eine Quote-Zeile von Apple (AAPL) k�nnte dabei folgenderma�en aussehen.

\begin{center}
\begin{tabular}{|c|c|c|c|}
\hline 
Bid & Bid Size & Ask & Ask Size \\ \hline
691.52 & 7 & 691.66 & 3 \\ \hline
\end{tabular}
\end{center}

Der Bid-Preis von \textdollar{691.52} ist das h�chste vorhandene Gebot f�r eine Apple-Aktie. Die Bid-Size gibt die Information �ber die Anzahl an Aktien, die K�ufer zu diesem Preis erstehen wollen. Die Anzahl wird in \emph{round lots} angegeben; meist entspricht ein Round Lot 100 St�ck der Aktie. Die Bid-Seite beschreit somit die K�uferseite. Der Ask-Preis und die Ask-Size beschreibt hingegen das beste Angebot auf der Verk�uferseite. In diesem Fall werden 3 Round Lots f�r den St�ckpreis von \textdollar{691,66} pro Aktie angeboten. Ein Round Lot w�rde in diesem Fall also \textdollar{69166,00} kosten.\\

Soll ein Handel schnellstm�glich abgewickelt werden, muss zum Ask-Preis gekauft und zum Bid-Preis verkauft werden. Wird in diesem Fall mindestens die Ask- bzw. Bid-Size gehandelt, ver�ndert sich die Quote-Zeile so, dass das n�chstbeste Angebot angezeigt wird. Diese Hintereinanderreihung von Angeboten wird auch als Orderbuchtiefe bezeichnet. \\

Angenommen, ein K�ufer ist nicht bereit, zum aktuellen Ask-Preis zu kaufen. Er will bessere Konditionen und schickt eine \emph{Limit-Order}, d.h. zu gegebenem Preis oder besser \footnote{IB-Ordertypen siehe: http://www.interactivebrokers.com/de/p.php?f=orderTypes}, mit einem Limit, das zwischen Ask- und Bid-Preis liegt. Sein Kaufgebot ist damit h�her als der zuvor h�chste und wird daher sofort in der Quote-Zeile auf der Bid-Seite angezeigt.\\

Der Aktienkurs wird mithilfe dieser vorhandenen Orders so gebildet, dass stets der h�chste Umsatz entsteht. Bei hoher Nachfrage auf K�uferseite steigt dadurch der Preis, dominieren jedoch die Verk�ufer sinkt der Preis. \cite{charttec-kursfeststellung}