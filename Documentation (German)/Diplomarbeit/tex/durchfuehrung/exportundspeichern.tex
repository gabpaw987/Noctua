\subsection{Export der Performancedaten}

Um die getesteten Algorithmen sowohl untereinander als auch �ber verschiedene Aktien-Preisdaten und Perioden ad�quat vergleichen und die Ergebnisse auswerten zu k�nnen, wurde der \gls{bts} die Funktionalit�t des Export der Performancedaten hinzugef�gt. Hierbei wird aus allen allgemeinen Performancedaten, die die gesamte Berechnung repr�sentieren, ein ausreichend formatiertes Text-File erstellt, das problemlos f�r den Menschen lesbar ist. Damit k�nnen die Ergebnisse verschiedener Algorithmen einfach ausgewertet werden, ohne die Berechnungen immer neu starten zu m�ssen. \\
\\
Nach einem Klick auf den "`Export"'-Button in der Men�leiste, kann man eine Datei ausw�hlen, in die die Daten exportiert werden sollen. Dabei wird jeder Wert des Performance-Tab, jedoch kein Order-spezifischer Wert exportiert.

\subsection{Speichern des Zustands der BTS}

Eine weitere Funktionalit�t, die die \gls{bts} bietet, ist das Speichern des aktuellen Zustands, der dann in der weiteren Folge auch wieder geladen werden kann. Hierbei werden alle zum momentanen Zeitpunkt gew�hlten Einstellungen auf dem Settings-Tab sowie alle Order-spezifischen und allgemeinen Performancedaten in eine Datei mit der Endung .bts gespeichert, um diesen Zustand sp�ter wieder herstellen zu k�nnen. Einzig das gezeichnete Chart wird nicht gespeichert, da dies in einem sehr gro�en Speicheraufwand resultieren k�nnte, da alle Aktien-Preisdaten der \gls{csv}-Datei dazu auch in die .bts-Datei gespeichert werden m�ssten. Falls sich die Pfadnamen allerdings nicht ver�ndert haben, kann man einfach die Berechnung nach dem Laden erneut starten und das Chart wird wieder gezeichnet. Man kann durch das Laden der Einstellungen sogar noch nachvollziehen, unter welchen Bedingungen die Berechnung damals durchgef�hrt wurde, also wie es zu den spezifischen Ergebnissen der Berechnung kam.\\
\\
Intern basiert das Speichern auf dem Prinzip der bin�ren Serialisierung (siehe \ref{Binaere-Serialisierung} "`Bin�re Serialisierung"') und das Laden somit auf der bin�ren Deserialisierung. Zum Speichern werden zuerst alle Order-spezifischen Performancedaten direkt serialisiert und in das gew�hlte File gespeichert. Anschlie�end werden alle allgemeinen Performancedaten in eine Liste mit dem Typ \inline{decimal} gespeichert. Danach wird diese serialisiert und unter die Oder-spezifischen Performancedaten in das File gespeichert. Genau die gleiche Prozedur wird auch f�r alle Einstellungen wiederholt, nur dass jeweils Listen mit den entsprechenden Datentypen zur Serialisierung benutzt werden. Zu guter Letzt wird noch die \inline{storeIndicatorStackPanels}-Methode (siehe \ref{Einstellungen-sichern} "`Persistente Sicherung der Einstellungen"') aufgerufen, um alle konfigurierten Indicator-\inline{StackPanel}s in eine \inline{StringCollection} zu serialisieren. Diese \inline{StringCollection} muss allerdings genauso bin�r serialisiert werden, um entsprechend in das .bts-File gespeichert werden zu k�nnen.\\
\\
Der Ladevorgang funktioniert grunds�tzlich genau gleich, nur dass eben die entsprechenden Methoden zur Deserialisierung aufgerufen und die einezlenen Werte wieder aus den Listen zur�ck in das ViewModel gespeichert werden m�ssen. Dabei wird zur Umwandlung der \inline{StringCollection} zur�ck in eine \inline{StackPanel}-Liste klarerweise die \inline{restoreIndicatorStackPanels}-Methode (siehe \ref{Einstellungen-sichern} "`Persistente Sicherung der Einstellungen"') aufgerufen. 