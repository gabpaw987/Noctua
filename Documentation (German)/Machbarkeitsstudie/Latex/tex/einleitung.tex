% !TEX root = ../Noctua_Machbarkeitsstudie.tex

\chapter{Einleitung} \label{chapter:einleitung}

Das algorithmische Handeln hat besonders in den letzten zwei Jahrzehnten rasant zugenommen. 2011 wurde davon ausgegangen, dass mindestens 30\% des Aktienhandelsvolumen in den USA bereits algorithmischer Natur ist. Alleine schon die Kommissionen der Broker f�r algorithmisch abgewickelte Entscheidungen beziehen sich weltweit auf etwa \textdollar{400-600} Millionen. Die meisten der verwendeten Algorithmen sind allerdings propriet�rer Natur und dienen h�ufig nicht der direkten Optimierung des Profits, sondern dem Aufteilen gro�er Orders und der Minimierung des Risikos und der Kosten. \cite{avellaneda-algorithmic-2011}\\
	Algorithmen, die direkt Handelsentscheidungen treffen, existieren ebenfalls und werden h�ufig f�r Fonds eingesetzt, die an Privatpersonen weiterverkauft werden. Besonders kleinere Unternehmen und Privatpersonen haben nicht die Kapazit�ten, um beim \gls{hft}, wo jede Millisekunde z�hlen kann, mitzumischen. Klassische Handelssysteme k�nnen allerdings sehr wohl als Computersoftware umgesetzt werden und bieten die Vorteile nach einem klar definierten System schneller Entscheidungen zu treffen und dies gegebenenfalls auch ohne menschliche Aufsicht.\\
	M�rkte, und insbesondere Aktienm�rkte, verhalten sich nicht immer invariant, sondern k�nnen ihre intrinsischen Systematiken mit der Zeit �ndern. Folglich kann es zu Einbu�en bei der Performance von Algorithmen kommen, die ebendiese Systematiken ausnutzen. Um dies zu verhindern oder zumindest die Effekte zu vermindern, muss ein erfolgreicher Algorithmus eine gewisse Adaption bzw. eine Anpassung und dadurch zus�tzliche Flexibilit�t der Parameter erm�glichen.
	Ziel des Projektes ist es, genau solch ein System zu entwickeln und umzusetzen. Die vollst�ndige Eigenentwicklung erm�glicht volle Einsicht in alle relevanten Bereiche des Algorithmus und der verwendeten Parameter. Au�erdem kann mithilfe einer zu entwickelnden \gls{bts} die Performance des Algorithmus anhand diverser Kriterien gezielt gemessen werden.
	Im folgenden geht es nun darum, zu kl�ren, welche Ans�tze zur Umsetzung eines solchen Algorithmus m�glich sind und wie ihre technische Implementierung aussehen k�nnte. Zus�tzlich soll die optimale L�sung sowohl f�r alle Aspekte des Algorithmus als auch der \gls{bts} er�rtert werden.