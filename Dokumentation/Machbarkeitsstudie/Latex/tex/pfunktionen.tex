\chapter{Produktfunktionen} \label{chapter:produktfunktionen}

\section{Must-Have}

/F010/\\
\textit{Vergangene Marktzust�nde bestimmen}

\begin{center}

\begin{tabular}{ | l | p{10cm} |}
\hline 
Beschreibung & Es sollen historische Marktzust�nde (innerhalb der letzten Jahre), auf transparenten
Aktienm�rkten, f�r die ein ausreichender Datenbestand vorhanden ist, automatisch bestimmt
werden. Sollten sich verschiedene gro�e M�rkte entgegen Erwartung entsprechend
unterschiedlich verhalten, dass diese keiner einheitlichen Analyse unterzogen werden k�nnen,
soll prim�r der US-amerikanische Aktienmarkt untersucht werden. Hierbei handelt es sich um
eine Gruppierung von Zeitabschnitten nach gemeinsamen Kriterien.\\  \hline
Aufwand & Hoch \\ \hline
Nutzen & Hoch \\ \hline
Ziel & Ermittelung von Klassen f�r Marktzust�nde. \\ \hline
Vorbedingungen & - \\ \hline
Nachbedingungen & - \\ \hline

\end{tabular}

\end{center}

/F020/\\
\textit{Aktuellen Marktzustand bestimmen}

\begin{center}

\begin{tabular}{ | l | p{10cm} |}
\hline 
Beschreibung & Dabei soll darauf geachtet werden, dass f�r eine fr�he Erkennung m�glicherweise nur ein Teil
der Daten vorhanden ist, die f�r die historische Analyse herangezogen werden.\\  \hline
Aufwand & Mittel \\ \hline
Nutzen & Hoch \\ \hline
Ziel & Zuordnung des aktuellen Marktzustandes zu einem bereits Bekannten. \\ \hline
Vorbedingungen & /F010/ Vergangene Marktzust�nde bestimmen \\ \hline
Nachbedingungen & - \\ \hline

\end{tabular}

\end{center}

/F110/\\
\textit{Trends erkennen}

\begin{center}

\begin{tabular}{ | l | p{10cm} |}
\hline 
Beschreibung & Durch \glspl{ma} soll es m�glich sein Trends in Aktienkursen zu identifizieren. Dazu
kommen verschiedene Crossover-Verfahren (double- / triple-crossover) oder Indikatoren, wie
der MACD (Moving Average Convergence Divergence) in Frage. Es soll eine statistisch
m�glichst profitable Variante hierf�r gefunden werden, die aufscheinende nachhaltige Trends
m�glichst g�nstig erkennt.\\  \hline
Aufwand & Hoch \\ \hline
Nutzen & Hoch \\ \hline
Ziel & Fr�hzeitige m�glichst profitable Erkennung von Trends. \\ \hline
Vorbedingungen & - \\ \hline
Nachbedingungen & - \\ \hline

\end{tabular}

\end{center}

/F120/\\
\textit{\gls{ma}-Dauer bestimmen}

\begin{center}

\begin{tabular}{ | l | p{10cm} |}
\hline 
Beschreibung & Je nachdem, wie lange ein Trend andauert, bedingt eine Trenderkennung andere \gls{ma}(-Paare)
mit unterschiedlichen Laufzeiten. Aus z.B. bew�hrten Wertepaaren oder adaptiven Methoden sollen
automatisch die optimalen Laufzeiten gew�hlt werden.\\  \hline
Aufwand & Niedrig \\ \hline
Nutzen & Mittel \\ \hline
Ziel & Erarbeitung eines optimalen Parametersatzes f�r ein \gls{ma}-Paar. \\ \hline
Vorbedingungen & - \\ \hline
Nachbedingungen & - \\ \hline

\end{tabular}

\end{center}

/F130/\\
\textit{An Marktzustand anpassen}

\begin{center}

\begin{tabular}{ | l | p{10cm} |}
\hline 
Beschreibung & Der Algorithmus soll sich durch Parameterver�nderung an den erkannten Marktzustand zur
Optimierung der Performance anpassen. Dies kann beispielsweise durch ver�ndern der \gls{ma}-Paare
oder durch Anpassung der Market Exposure und damit des Risikos erfolgen.
Dazu \textit{k�nnen} die Implikationen durch Nachforschung bekannt sein, woraufhin ein Modell
angewandt wird, m�ssen aber nicht, da auch induktiv aus den Implikationen gelernt werden
kann, wonach automatisch ein Modell entsteht. (\textit{Maschinelles Lernen}) Dabei werden f�r die
unterschiedlichen Markzust�nde verschiedene Parameters�tze durchprobiert.\\  \hline
Aufwand & Hoch \\ \hline
Nutzen & Hoch \\ \hline
Ziel & Anpassung der Hauptfunktionen des Algorithmus an den aktuellen Marktzustand. \\ \hline
Vorbedingungen & /F020/ Aktuellen Marktzustand bestimmen \\ \hline
Nachbedingungen & - \\ \hline

\end{tabular}

\end{center}

/F140/\\
\textit{Signale generieren}

\begin{center}

\begin{tabular}{ | l | p{10cm} |}
\hline 
Beschreibung & Signalgeben bei potentiellen Einstiegspunkten (long signal) und Ausstiegspunkten (short
signal).\\  \hline
Aufwand & Niedrig \\ \hline
Nutzen & Hoch \\ \hline
Ziel & R�ckgabe von Handelssignalen. \\ \hline
Vorbedingungen & /F110/ Trends erkennen, /F130/ An Marktzustand anpassen, /F120/ \gls{ma}-Dauer bestimmen \\ \hline
Nachbedingungen & /F160/ Signale filtern \\ \hline

\end{tabular}

\end{center}

/F150/\\
\textit{Trend-Nachhaltigkeit bestimmen}

\begin{center}

\begin{tabular}{ | l | p{10cm} |}
\hline 
Beschreibung & Durch geeignete Support- und Resistance-Indikatoren soll die Nachhaltigkeit eines Trends
bestimmt werden (beispielsweise Pivot Points, RSI, CCI oder MAs), um den Ausstiegspunkt zu
optimieren.\\  \hline
Aufwand & Mittel \\ \hline
Nutzen & Hoch \\ \hline
Ziel & Festellen der Nachhaltigkeit erkannter Trends. \\ \hline
Vorbedingungen & /F140 Signale generieren \\ \hline
Nachbedingungen & - \\ \hline

\end{tabular}

\end{center}

/F160/\\
\textit{Signale filtern}

\begin{center}

\begin{tabular}{ | l | p{10cm} |}
\hline 
Beschreibung & Zur Verminderung von unprofitablen, zu kurzen Trades sollen insbesondere Kaufsignale
gefiltert werden. Die Trenderkennung k�nnte des �fteren zu kurz anhaltende Trends
erkennen, indem beispielsweise ein MA-Crossover nur f�r kurze Zeit besteht. Durch das
Einf�hren eines Schwellenwertes (threshold), der �berschritten werden muss, oder eine
bestimmte Zeitspanne, die ein Signal �berdauern muss k�nnen zu kurze Trades vermindert
werden, wenn sich im Backtesting dadurch ein Vorteil herausgestellt hat.\\  \hline
Aufwand & Mittel \\ \hline
Nutzen & Hoch \\ \hline
Ziel & Filterung der unbrauchbaren Signale. \\ \hline
Vorbedingungen & /F140/ Signale generieren, /F150/ Trend-Nachhaltigkeit bestimmen \\ \hline
Nachbedingungen & - \\ \hline

\end{tabular}

\end{center}

/F210/\\
\textit{Performance berechnen}

\begin{center}

\begin{tabular}{ | l | p{10cm} |}
\hline 
Beschreibung & Die relative Performance eines Algorithmus soll in Prozent der Kapitalver�nderung berechnet
werden.\\  \hline
Aufwand & Mittel \\ \hline
Nutzen & Hoch \\ \hline
Ziel & Berechnung der relativen Performance eines Algorithmus \\ \hline
Vorbedingungen & - \\ \hline
Nachbedingungen & - \\ \hline

\end{tabular}

\end{center}

/F220/\\
\textit{Gewinn/Risiko-Verh�ltnis berechnen}

\begin{center}

\begin{tabular}{ | l | p{10cm} |}
\hline 
Beschreibung & Bestimmung des Risikos des Algorithmus (beispielsweise anhand der Volatilit�t) in
Verbindung mit der Performance (e.g. sharpe ratio).\\  \hline
Aufwand & Niedrig \\ \hline
Nutzen & Mittel \\ \hline
Ziel & Bestimmung des Gewinn/Risiko-Verh�ltnisses eines Algorithmus \\ \hline
Vorbedingungen & /F210/ Performance berechnen \\ \hline
Nachbedingungen & - \\ \hline

\end{tabular}

\end{center}