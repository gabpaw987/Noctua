%%-----------------------------------------------------------------------
% Loading the packages and classes
%%-----------------------------------------------------------------------

\documentclass[12pt,twoside,a4paper,final]{book}%
\usepackage{a4}%                        %% Verwendet mehr Platz auf einer A4 Seite als Option a4paper in book
\usepackage[english,ngerman]{babel}%    %% Babel Sprachen
\usepackage[latin1]{inputenc}%          %% input encoder f�r umlaute usw.
\usepackage[T1]{fontenc}
\usepackage{amssymb}%                   %% AMS Symbole
\usepackage{amsmath}%                   %% AMS Math Funktionen
\usepackage{amsfonts}%
\usepackage[Sonny]{fncychap}%           %% Chapter Style
\usepackage[english,noprefix]{nomencl}%  %% Nomenclature (Symbolverzeichnis)
\usepackage{makeidx}%                   %% Index
\usepackage{graphicx}%                  %% Graphiken
\graphicspath{{img/}}
\DeclareGraphicsExtensions{.pdf,.jpeg,.png,.jpg}
\usepackage{psfrag}%                    %% Tex-Schriftarten und Formeln in EPS-Grafiken
\usepackage{color}
\usepackage{nicefrac}
\usepackage{ifthen}
\usepackage{fancyhdr}
\usepackage{subfigure}
\usepackage[xindy, acronym, toc]{glossaries}
\usepackage{cite}
\usepackage{listings}
\usepackage{color}
\usepackage[dvipsnames]{xcolor}
\usepackage{rotating}
\usepackage{multirow}
\usepackage{eurosym}
%Define Acronyms like
% use on every place in your document \gls{mas} for TGM or - for plural - use \glspl for TGMs
% at the first usage of this, the acronym will be introduced, everywhere else it will only be the in the short form: ``Technologisches Gewerbemuseum (TGM)''
% TIPP: USE THIS FOR EVERY NAME/SOFTWARE-TOOL/MAIN PART OF YOUR WORK, like JAVA, - so that, e.g. JAVA is not written Java everywhere else in your thesis.

\newacronym{tgm}{TGM}{Technologisches Gewerbemuseum}
\newacronym{fp}{FP-Analyse}{Function-Point-Analyse}
\newacronym{pm}{PM}{Personenmonate}

\newacronym{ma}{MA}{Moving Average}
\newacronym{sma}{SMA}{Simple Moving Average}
\newacronym{lwma}{LWMA}{Linear Weighted Moving Average}
\newacronym{wma}{WMA}{Linear Weighted Moving Average}
\newacronym{ema}{EMA}{Exponential Moving Average}
\newacronym{ama}{AMA}{Adaptive Moving Average}
\newacronym{tma}{TMA}{Triangular Moving Average}
\newacronym{er}{ER}{Efficiency Ratio}
\newacronym{dema}{DEMA}{Double Exponential Moving Average}
\newacronym{tema}{TEMA}{Triple Exponential Moving Average}
\newacronym{adx}{ADX}{Average Directional Movement Index}
\newacronym{sf}{SF}{Smoothing Factor}
\newacronym{macd}{MACD}{Moving Average Convergence/Divergence}
\newacronym{cci}{CCI}{Commodity Channel Index}
\newacronym{rsi}{RSI}{Relative Strength Indicator}

\newacronym{forex}{FOREX}{Foreign Exchange Market}
\newacronym{hft}{HFT}{High Frequency Trading}

\newacronym{bts}{BTS}{Backtesting-Software}
\newacronym{git}{GIT}{GIT-Server}
\newacronym{ide}{IDE}{Integrated Development Environment}
\newacronym{mql5}{MQL5}{MetaQuotes Language 5}
\newacronym{gui}{GUI}{Graphical User Interface}
\newacronym{dll}{DLL}{Dynamic Link Library}
\newacronym{xml}{XML}{eXtensible Markup Language}
\newacronym{xaml}{XAML}{eXtensible Application Markup Language}
\newacronym{clr}{clr}{Common Language Runtime}
\newacronym{cls}{cls}{Common Language Specification}
\newacronym{wpf}{WPF}{Windows Presentation Foundation}
\newacronym{jvm}{JVM}{Java Virtual Machine}
\newacronym{api}{API}{Application Programming Interface}
\newacronym{mvvm}{MVVM}{Model View Viewmodel}
\newacronym{soap}{SOAP}{Simple Object Access Protocol}
\newacronym{linq}{LINQ}{Language Integrated Query}
\newacronym{sql}{SQL}{Structured Query Language}
\newacronym{csv}{CSV}{Comma-separated values}

\newacronym{afa}{AFA}{Abschreibung f�r Abnutzung}

\newglossaryentry{DLL}{
	name=DLL,
	description={Dynamic Link Library (DLL) bezeichnet allgemein eine dynamische Programmbibliothek, meist bezieht sich der Begriff jedoch auf die von Microsoft f�r die Betriebssysteme Microsoft Windows und OS/2 verwendete Variante.}
}
\newglossaryentry{F-Sharp}{
	name=F-Sharp,
	description={F\# ist eine funktionale Programmiersprache von Microsoft f�r das .NET-Framework.}
}
\newglossaryentry{BTS}{
	name=Backtesting-Software,
	description={Als BTS wird die im Laufe dieses Projekts erstellte Backtesting-Software bezeichnet, die einen Algorithmus in Form einer DLL-Datei auf seine Performance �ber historische Aktien-Preisdaten testen kann.}
}
\newglossaryentry{CSV}{
	name=Comma-separated values,
	description={Das Dateiformat CSV steht f�r englisch Comma-separated values (seltener Character-separated values) und beschreibt den Aufbau einer Textdatei zur Speicherung oder zum Austausch einfach strukturierter Daten. Die Dateinamenserweiterung lautet .csv.}
}
\newglossaryentry{Bar}{
	name=Bar,
	description={Im Aktienhandel versteht man unter einem Bar die Zusammenfassung aller Preis�nderungen eines Instruments �ber einen festgelegten Zeitraum. Ein Bar wird durch die Werte Open (erster Wert), High (h�chster Wert), Low (niedrigster Wert), Close (letzter Wert) innerhalb eines bestimten Zeitraums definiert.}
}
%%-----------------------------------------------------------------------
% Using the Hyperref-Package for PDF-Online Version
%%-----------------------------------------------------------------------

\def\usehyperref{1}

\ifnum\usehyperref=1
\usepackage[pdftex=true,
  pdftitle={Diplomarbeit},
  pdfauthor={Gottfried Koppensteiner},
  bookmarksopen,
  colorlinks,
  citecolor=blue,
  linkcolor=blue,
  breaklinks%
]{hyperref}%
\fi



%%-----------------------------------------------------------------------
% Rearranging Nomenclature
%%-----------------------------------------------------------------------


\renewcommand{\nomname}{List of symbols}
%\renewcommand{\nompreamble}{The following list only contains symbols
%  that are used continuously throughout the text. Local symbols are
%  not listed.}
\renewcommand{\nomgroup}[1]{
 \ifthenelse{\equal{#1}{A}}{\item[\textbf{General symbols}\bigskip]}{
 \ifthenelse{\equal{#1}{B}}{\item[\bigskip\bigskip\textbf{Chapter 2}\bigskip]}{
 \ifthenelse{\equal{#1}{C}}{\item[\bigskip\bigskip\textbf{Chapter 3}\bigskip]}{
 \ifthenelse{\equal{#1}{D}}{\item[\bigskip\bigskip\textbf{Chapter 4}\bigskip]}{
 \ifthenelse{\equal{#1}{E}}{\item[\bigskip\bigskip\textbf{Chapter 5}\bigskip]}{
 \ifthenelse{\equal{#1}{F}}{\item[\bigskip\bigskip\textbf{Appendix}\bigskip]}{
 }}}}}}}
\makenomenclature



%%-----------------------------------------------------------------------
% Makes Bibliography available in Winedt
%%-----------------------------------------------------------------------

%GATHER{bib_kiefer.bib}


%%-----------------------------------------------------------------------
% Definition of possible environments
%%-----------------------------------------------------------------------

\newtheorem{theorem}{Theorem}[chapter]
\newtheorem{acknowledgement}{Acknowledgement}[chapter]
\newtheorem{algorithm}{Algorithm}[chapter]
\newtheorem{axiom}{Axiom}[chapter]
\newtheorem{case}{Case}[chapter]
\newtheorem{claim}{Claim}[chapter]
\newtheorem{conclusion}{Conclusion}[chapter]
\newtheorem{condition}{Condition}[chapter]
\newtheorem{conjecture}{Conjecture}[chapter]
\newtheorem{corollary}{Corollary}[chapter]
\newtheorem{criterion}{Criterion}[chapter]
\newtheorem{definition}{Definition}[chapter]
\newtheorem{example}{Example}[chapter]
\newtheorem{exercise}{Exercise}[chapter]
\newtheorem{lemma}{Lemma}[chapter]
\newtheorem{notation}{Notation}[chapter]
\newtheorem{problem}{Problem}[chapter]
\newtheorem{proposition}{Proposition}[chapter]
\newtheorem{remark}{Remark}[chapter]
\newtheorem{solution}{Solution}[chapter]
\newtheorem{summary}{Summary}[chapter]
\newenvironment{proof}[1][Proof]{\noindent\textbf{#1.} }{\ \rule{0.5em}{0.5em}}

\renewcommand{\chaptermark}[1]{\markboth{\thechapter.\ #1}{}}
\renewcommand{\sectionmark}[1]{\markright{\thesection.\ #1}}




%%-----------------------------------------------------------------------
% Marking of overfull boxes and increasing of tolerances
%%-----------------------------------------------------------------------

% F�r die Final-Version die n�chste Zeile auskommentieren um schawarze Balken (TU-Logo im Titelblatt) zu ignorieren!
\overfullrule=10pt%                     %% Markiert �berf�llte Boxen. z.b. hbox overfull (Evtl. nicht im pdf sichtbar!!!!)
\hfuzz=1pt%                             %% Toleranz bei hbox overfull erh�ht 1pt entspr. ca. 1/3 mm


%%-----------------------------------------------------------------------
% Counter
%%-----------------------------------------------------------------------


\setcounter{secnumdepth}{3}%
\setcounter{tocdepth}{3}%



% Clear Header Style on the Last Empty Odd pages
\makeatletter
\def\cleardoublepage{\clearpage\if@twoside \ifodd\c@page\else%
    \hbox{}%
    \thispagestyle{empty}%              % Empty header styles
    \newpage%
    \if@twocolumn\hbox{}\newpage\fi\fi\fi}
\makeatother

%%-----------------------------------------------------------------------
% Avoid indents
%%-----------------------------------------------------------------------

\setlength{\parindent}{0pt}

%%-----------------------------------------------------------------------
%% Hyphenation for german abstract
%%-----------------------------------------------------------------------

\hyphenation{Fa-mi-lie
             Ski-bil-dung
             Ar-beits-wal-ze
             neg-lec-ted
             se-par-ate
             di-men-sio-nal
             her-r\"uhren
             N\"a-herungs-l\"os-ungen
             wissen-schaft-licher
             Regelungs-technik
             re-con-fi-gur-abili-ty
             manage-ment
             manu-facturing
             not-wendigen
             }


%%-----------------------------------------------------------------------
% Colored grafix
% 1 = color
% 0 = grey
%%-----------------------------------------------------------------------


\def\colorsw{1}

%%-----------------------------------------------------------------------
% Additional remarks
% 1 = with remarks
% 0 = without remarks
%%-----------------------------------------------------------------------


\def\addnotes{0}


%%-----------------------------------------------------------------------
% Define month
%%-----------------------------------------------------------------------

\def\monthdis{Oktober 2012}

\makeglossaries
%%-----------------------------------------------------------------------
% Document
%%-----------------------------------------------------------------------


\begin{document}%
\selectlanguage{ngerman}%
\renewcommand{\indexname}{Index}%
\topmargin15.0mm


\def\tpdefault{{\sf \center \vspace*{-4cm}
%\begin{center}
%\hspace*{-1.3cm}
%\rule{17cm}{0.02cm}
%\end{center}


\begin{figure}[h]
\begin{flushright}	
		\includegraphics[width=0.3\textwidth]{graphics/title/tgmlogo2.png}
	\label{fig:tgmlogo}
\end{flushright}
\end{figure}


\vspace{2cm}


{\Large %\bf 
Pflichtenheft\\ \vspace{0.7cm}}
 {\LARGE \sloppy
{\bf \sf  \textbf{NOCTUA \\}
Wertpapierhandelsalgorithmus mit\\
Marktzustandsadaption
\\}}
%
%
\vspace*{2cm}
{\normalsize Ausgef\"uhrt in Zuge des Projektmanagement-Unterrichts im 5. Jahrgang\\
Ausbildungszweig Systemtechnik/Medientechnik\\ %unzutreffendes streichen
  \vspace{1.5cm}
  \normalsize unter der Leitung von\\
  \large Prof.\ Mag.\ Hans Brabenetz\\
  \normalsize Abteilung f�r
  Informationstechnologie\\
  \vspace{1.5cm}
  eingereicht am  Technologischen Gewerbemuseum Wien\\
  H\"ohere Technische Lehr- und Versuchsanstalt\\
  Wexstrasse 19-23, A-1200 Wien\\
  }}}


\begin{titlepage}
	\tpdefault
	{\sf \center \vspace{1.0cm}
	\normalsize von\\
	\large 
	Peer Nagy 5CHITI\\
	Gabriel Pawlowsky, 5BHITS\\
	Josef Sochovsky, 5BHITS\\
	\vspace {2 cm}
	\bf \sf {Wien, im \monthdis} \\
		%	\vspace{2cm}
	%	\rule{\textwidth}{0.01cm}
	
	}



	\end{titlepage}
\frontmatter%   %% front matter will be numbered in small Roman letters

%%-----------------------------------------------------------------------


%TCIDATA{OutputFilter=latex2.dll}
%TCIDATA{Version=5.00.0.2552}
%TCIDATA{LaTeXparent=0,0,Dissertation_SW.tex}


\chapter*{Vorwort}

Diese Arbeit wurde im Jahr 2012/13 im Zuge unserer Ausbildung in der Abteilung f�r Informationstechnologie am \gls{tgm}, HTBLVA Wien 20, durchgef�hrt. 


\bigskip

Dankesworte

\bigskip
\bigskip
\bigskip
\bigskip



Wien, im \monthdis \hfill Nagy, Pawlowsky, Sochovsky \vfill
%
\selectlanguage{english}%
\chapter*{Abstract}

This is the english abstract.%
\selectlanguage{ngerman}%
\include{tex/kurzzfsg}%


%%-----------------------------------------------------------------------
% Define Header for Content chapter
%%-----------------------------------------------------------------------


\makeatletter
\def\tableofcontents{\chapter*{\contentsname\@mkboth{\contentsname}{\contentsname}}
  \@starttoc{toc}}
\makeatother

\clearpage%
\tableofcontents
\clearpage
\listoffigures
\clearpage
\lstlistoflistings %\listoftables
\clearpage 
\markboth{Contents}{Contents}


%\addcontentsline{toc}{chapter}{\numberline{}\listfigurename}%
%\listoffigures
%\listoftables%
%\addcontentsline{toc}{chapter}{\numberline{}\listtablename}%
\clearpage%



\nomenclature[aa]{$t$}{time}
\nomenclature[bb]{$t_0$}{reference time}
\nomenclature[aa]{$m$}{mass}
\nomenclature[aa]{$\rho$}{mass density}

\markboth{\nomname}{\nomname}%
\addcontentsline{toc}{chapter}{\numberline{}\nomname}%
\printnomenclature


\mainmatter%   %% main part will be numbered in Arabic letter


% include chapters
% Chapter1
\chapter{Einf�hrung} \label{chapter:einfuehrung}

\section{Hintergrund und Motivation} \label{hintergrund}

\subsection{Hintergrund}

Seit dem Beginn des 17. Jhdt., wo die ersten Anteilsscheine von Unternehmen ausgegeben wurden
hat sich der Finanzmarkt in Riesenschritten vorw�rts bewegt und ist heutzutage eine Multi-
Milliarden-Dollar Industrie. Aktien sind l�ngst nicht mehr die einzigen Wertpapiere, die auf dem
Finanzmarkt gehandelt werden. Instrumente wie Optionsscheine oder Future-Contracts sind dabei
noch etabliertere Handelsg�ter.\\
\\
Der Handel mit Wertpapieren ist in den letzten Jahren und Jahrzehnten zunehmend systematisiert
und automatisiert worden. Kaum jemand trifft Handelsentscheidungen leichtfertig aus dem Bauch
heraus ohne fundierte Analyse. Diese Analyse unterwirft sich aber damit einem programmatischen
Schema, das ebenso gut auch automatisch, algorithmisch angewandt werden kann: trifft ein Mensch
Entscheidungen nach einem genauen Schema, kann ein Computer dies ebenso und dabei sogar
schneller und genauer.\\
Besonders gut geeignet daf�r scheinen die technische Analyse, besonders die Trendbestimmung und
das Trendfolgen. Auch wenn es reichlich Kritik an solchen Systemen gibt (besonders, darauf
basierend, dass sich Aktienkurse nach keiner bekannten statistischen Verteilung bewegen), wenden
sehr viele Marktteilnehmer solche Systeme an. Das f�hrt zumindest teilweise aber zu einer
selbsterf�llenden Prophezeiung, da sich die Kurse am Verhalten der Majorit�t der Marktteilnehmer
orientieren.\\
Ein weiterer Vorteil des Algorithmischen Trading ist die Geschwindigkeit, sowie Genauigkeit mit
der Computer arbeiten k�nnen, an die Menschen nicht heranreichen. Durch systematische und
statistische Entscheidungen k�nnen menschliche Emotionen aus dem Spiel gelassen und dadurch
auch das Risiko besser abgesch�tzt werden.\\
\\
Die Informationsflut, die schon f�r normale Aktien existiert, sind l�ngst nicht mehr manuell zu
bew�ltigen. R�ume voller Server rechnen ununterbrochen mit den Kursdaten und versuchen jeden
m�glichen Vorteil auszunutzen, um den Gewinn zu optimieren. Die verwendeten Modelle, Ans�tze
und Algorithmen werden in der Regel geheim gehalten, da viel Geld von Entscheidungen der
Software abh�ngt. W�ren verwendete Algorithmen publik, w�rden diese nicht mehr lange profitabel
funktionieren. Ergo bleiben die Vorgangsweisen, die besser funktionieren und gro�en Unternehmen
reale Gewinne einbringen geheim. Im Internet gibt es viele Quellen, die verschiedene Ans�tze
erkl�ren, wie mit vorhandenen Daten Handelsentscheidungen berechnet werden k�nnen, die in
Tradingsoftware implementiert werden k�nnten. Diese Standardalgorithmen sind au mass
vorhanden, ein objektiver Vergleich ist aber entweder sehr umst�ndlich oder gar nicht m�glich. Ein
besonderer Fokus dieses Projektes soll daher darauf liegen entwickelte Algorithmen auf ihre
Performance zu pr�fen. Zu diesem Zweck wird eine \gls{bts} programmiert, welche
die Signale eines Algorithmus f�r einen historischen Datenbestand generiert und die resultierenden
virtuellen Trades analysiert.\\
\\
Momentan haben sowohl B�rseninteressierte als auch professionelle Trader ein Problem damit, die
Performance unterschiedlicher Algorithmen ad�quat zu klassifizieren und damit auch zu
vergleichen. Das riesige Spektrum an algorithmischer Tradingsoftware, das derzeit angeboten wird,
ist zudem auch noch sehr undurchsichtig, da es sich bei fast allen arbeitenden Algorithmen um
closed Source Produkte handelt. Der normale Benutzer kann beim Kauf einer solchen Software also nur die
Kompetenz des anbietenden Unternehmens einsch�tzen und hat keine schriftliche und �berpr�fbare
Grundlage, die die Qualit�t des Algorithmus aufzeigt, in dessen H�nde er sein Geld legt.\\
\\
Am einfachsten kann hierbei durch die Entwicklung eines eigenen oder durch den Kauf des Source
Codes eines anderen Algorithmus Abhilfe geschafft werden. Ohne die Noctua-\gls{bts}
w�re es jedoch trotzdem nicht m�glich, solche Algorithmen auf ihre Performance oder auch auf die
Arbeitsweise w�hrend unterschiedlichen Marktzust�nden zu testen.\\
Zus�tzlich fixiert sich die absolute Mehrheit der derzeit vorherrschenden Algorithmen zu stark auf
die reine Kursentwicklung und beachtet nicht den aktuellen Marktzustand, obwohl dieser einer der
wichtigsten Grundsteine zur Vorhersage der Kurse ist.\\
Nat�rlich ist es au�erdem zurzeit nicht m�glich sein Kapital so fixverzinst anzulegen, dass man
einen Jahreszinssatz von 10\% erh�lt. Am n�chsten kann man dem nur durch die sehr risikoarme
Verwaltung seines Kapitals durch einen ausgekl�gelten Algorithmus kommen, der genau dies
verspricht.

\subsection{Motivation}

F�r ein automatisiertes Handeln mit Wertpapieren an der B�rse soll ein Algorithmus
entwickelt werden, der die optimalen Handelsentscheidungen berechnet. Damit man den
fertigen Algorithmus, sowie dessen Vorg�ngerversionen, als auch andere Standartalgorithmen
(wie zum Beispiel: ) miteinander vergleichen kann, ist es erforderlich eine Software zu
implementieren die automatisch mit historischen Daten, ergo bereits vergangene Bewegungen
an der B�rse, rechnet und eine vern�nftige Testumgebung erzeugt.\\
\\
Die Bestandteile des Projektes werden in 3 Bereiche gegliedert:
\begin{itemize}
	\item Der Algorithmus soll Trends m�glichst fr�h identifizieren und diesen solange folgen
				bis sie durch Support- und Resistance-Level ihre Nachhaltigkeit verlieren.
				Diese Identifikation soll, mithilfe historischer Daten funktionieren. Damit mit man die
				Qualit�t der Algorithmen in verschiedenen Marktzust�nden die jeweilige Qualit�t der
				Berechnungen feststellen kann, ist es notwendig den Markttrend zu gewissen Zeiten
				bereits zu kennen. Das bedeutet, dass die vorhandenen Daten analysiert werden
				m�ssen, um die verschiedenen Marktumschw�nge und deren Auswirkungen auf den
				Algorithmus zu erkennen und auslesen zu k�nnen.
	\item Um die verschiedenen Kursmuster bei diversen Marktstimmungen in den Algorithmus
				zu integrieren, unterscheidet dieser zwischen Marktzust�nden, die durch
				gesamtwirtschaftliche Zusammenh�nge (W�hrungswechselkurse, Commodity-Preise,
				Leitindizes, u.�.) bestimmt werden k�nnen. Hierbei muss besonders vorsichtig mit
				den �nderungen an dem Algorithmus vorgegangen werden, sonst kann es leicht
				passieren, dass eine kleine �nderung der Grundberechnung die Entscheidung bei einer
				v�llig unerwarteten Trend�nderung beeintr�chtigt. Deswegen ist es wichtig jede
				�nderung zu dokumentieren und zu versionieren. Dies soll einerseits das Endergebnis
				der Entwicklung verbessern und qualitativ hochwertiger halten und andererseits eine
				signifikante Geschwindigkeitssteigerung in der Algorithmuskodierung erzeugen.
	\item Damit die Performance auch �ber mehrere Versionen des Algorithmus verglichen
				werden kann, soll eine \gls{bts} entwickelt werden, diese wird mithilfe
				der historischen Daten funktionieren. Dieses Programm simuliert einen rapiden Ablauf
				von m�glicherweise mehreren Jahren B�rsengeschichte, die den Rechenablauf mit
				verschiedenen Marktzust�nden und Marktzustands�nderungen konfrontiert. Die
				Ergebnisse dieser Simulation k�nnen dann ausgelesen werden, um sie mit den
				vorherigen Ergebnissen auf eine etwaige Steigerung oder auch einen Abfall der
				Performance zu analysieren. Damit man einen reibungslosen Austausch der
				verschiedenen selbst entwickelten Rechenmodulen, aber auch der vorhandenen
				Standardalgorithmen anbieten kann, muss daf�r gesorgt werden, dass das simple und
				effiziente Austauschen des momentan verwendeten Algorithmus erm�glicht wird.
\end{itemize}

Noctua soll es vereinfachen verschiedene Ideen, die teils auf dubiosen Webseiten angepriesen
werden, mit einander vergleichbar zu machen und den m�glichen Gewinn und das Risiko zu
klassifizieren. Dazu muss nur noch der Algorithmus nach vorgegebener Art und Weise
implementiert und von der \gls{bts} getestet werden.\\
Durch die Entwicklung des Noctua-Algorithmus werden au�erdem eine Reihe von Ideen
ausgetestet, wodurch ebenfalls die Performance der angewendeten Indikatoren und damit
verbundenen Tradingstrategien verifiziert werden kann. Zus�tzlich entsteht durch das
inkrementelle Voranschreiten der Algorithmus-Version objektiv messbares Wissen �ber
finanzwirtschaftliche Zusammenh�nge im Bereich des algorithmischen Trading.


%
% Chapter2

\chapter{Machbarkeitsstudie} \label{chapter:machbarkeitsstudie}



%
% INCLUDES (\input) FOR CHAPTER 2 SUBFILES MOVED TO chapter2.tex
%\section{Voruntersuchung}\label{section:voruntersuchung}

\subsection{IST-Erhebung}

Es wird bereits ein Gro�teil des Kapitals in Wertpapieren algorithmisch verwaltet. Daher existiert eine Unmenge an Wissen �ber den Aufbau
von B�rsenalgorithmen und die Anwendung von Indikatoren zur Einsch�tzung von zuk�nftigen Kurswerten. Die meisten dieser Algorithmen basieren auf technischer
Analyse und den simplen Indikatoren, die diese mit sich bringt. Bekannte Vertreter davon sind zum Beispiel der \gls{macd} und der \gls{cci}.
Die meisten Algorithmen benutzen au�erdem eine Zusammensetzung aus verschiedenen \glspl{ma}, um die zu Grunde liegenden Handelsentscheidungen zu treffen.
Aufgrund dieses umfangreichen Wissens kann man weitere M�glichkeiten erforschen und noch besser und sinnvoller handelnde maschinelle Helfer kreieren.
Weniger umfangreich ist allerdings das Wissen �ber Marktzust�nde. Es gibt eine Menge Aufzeichnungen �ber die simplen Marktphasen(Aufw�rts-, Abw�rts-, Seitw�rtstrend), doch Methoden zur Kategorisierung des Marktes in neue Klassen sind eher wenig vorhanden bzw. schlecht oder nur unternehmensintern zug�nglich.\\
\\
Ein Problem der aktuellen Situation ist allerdings das schlechte bzw. umst�ndliche Testing dieser Algorithmen, da wenig Software existiert, die
verifizieren kann, ob ein Algorithmus in bestimmten Marktphasen bestimmte Leistungen erbringt. Au�erdem ist es momentan ziemlich kompliziert, sich einfach
die gesamte Performance eines solchen Handelsalgorithmus anzusehen. Es gibt hierf�r aber sowohl Gratisquellen als auch kommerzielle Produkte, von denen man
historische Daten zum Backtesting dar Algorithmen beziehen kann. Das Projektteam von Noctua besitzt bereits ca. 3.9 GB an historischen B�rsenkursen von
e-Signal\footnote{http://www.esignal.com/default.aspx?tc=} und nahezu unbegrenzten Zugriff auf weitere Daten vom selben Anbieter.
Dadurch ist es ihm m�glich, Algorithmen �ber ein weites Spektrum von Marktphasen und Marktzust�nden hinweg zu testen und die Performance verschiedener Algorithmen in unterschiedlichen Situationen akkurat festzustellen. Dies ist besonders wichtig, da Algorithmen die in der nahen Vergangenheit gute Entscheidungen trafen, meist weiterhin sehr erfolgreich handeln und den Anleger m�glicherweise mit einem Kapitalzuwachs belohnen.

\subsection{IST-Zustand}

Aufgrund der riesigen Industrie die diesem Projekt zu Grunde liegt gibt es nat�rlich bereits eine Vielzahl an Konkurrenzprodukten auf dem Markt. Einige davon spezialisieren sich auf das Backtesting bzw. die Bereitstellung einer Plattform zur Entwicklung von Algorithmen. Andere sind eher propriet�rer Natur und versuchen lediglich durch Korruption der Konkurrenz selbst den wirtschaftlich rentabelsten Algorithmus zu betreiben. Doch alle haben ihre Vor- und Nachteile. Daher finden sich auch immer wieder neue Nischen f�r Neueinsteiger am Markt, die durch ausgekl�gelte Algorithmen viel erreichen k�nnen. Im folgenden werden nun die bekanntesten dieser Konkurrenzprodukte beschrieben.

\subsubsection{MetaTrader 5}

Hierbei\footnote{http://www.metatrader5.com/} handelt es sich um ein ziemlich umfangreiches, ein wenig �lteres Produkt, dass Futures, Options und Aktiendaten anbietet, aber eigentlich auf \gls{forex}-Daten spezialisiert ist. MetaTrader 5 ist die neue verbesserte Version von Metatrader 4 und bietet neben den alten Funktionen Neuerungen wie die Einbindung von News. Au�erdem bietet der MetaTrader ein weites Spektrum an Indikatoren, die einfach in den laufenden Betrieb eingebunden werden k�nnen. Dieses wurde mit der Version 5 auch noch weiter vergr��ert, was viele langj�hrige Nutzer ausdr�cklich loben. Die Oberfl�che des MetaTraders sieht in etwa aus, wie in Abbildung \ref{fig:metatrader5_overview} dargestellt.

\begin{figure}
	\centering
		\includegraphics{graphics/chapter2/metatrader5_overview.png}
	\caption{MetaTrader 5 Oberfl�che}
	\label{fig:metatrader5_overview}
\end{figure}

\begin{figure}
	\centering
		\includegraphics[width=1.00\textwidth]{graphics/chapter2/metaeditor_robot_modification.jpg}
	\caption{MetaTrader 5 IDE-Oberfl�che}
	\label{fig:metaeditor_robot_modification.jpg}
\end{figure}

Doch das gr��te Feature des MetaTraders ist die eingebaute IDE (siehe Abbildung \ref{fig:metaeditor_robot_modification.jpg}), die es mittels einer eigenen MetaTrader-spezifischen Sprache, der \gls{mql5}, erm�glicht, eigene Algorithmen zu programmieren und diese dann auch direkt in den laufenden Betrieb zu �bernehmen. Der MetaTrader bietet sogar einen j�hrlichen Wettbewerb an, bei dem er jedes Jahr einen der besten Algorithmen zum Sieger k�rt. Die neue Sprache \gls{mql5} ist sogar ebenfalls noch weiter gegen�ber der MQL4 verbessert. Dies f�hrt uns allerdings auch schon zum Problem beim MetaTrader 5, da nur ein geringer Teil der Softwareentwickler gewillt ist, wirklich eine neue Programmiersprache zu lernen, nur um einen Algorithmus entwickeln und testen zu k�nnen, der dann selbst wiederum auch an den MetaTrader gebunden ist nicht in den normalen operativen Betrieb portiert werden kann. Au�erdem ist der MetaTrader eines der sehr wenigen Produkte am Markt, die es Nutzern wirklich erm�glicht einen Algorithmus zu entwickeln und zu testen ohne die gesamte Struktur rundherum zuerst aufzubauen. Daher ist es n�tig diesem Mangel Abhilfe zu schaffen und eine \gls{bts} zu entwickeln, die genau dies erm�glicht, ohne den Nutzer dabei an das Unternehmen zu binden, in dem er seinen Algorithmus testet.

\subsubsection{Algorithmen}

Es existieren wie bereits des �fteren erw�hnt Unmengen an propriet�ren Algorithmen, die Zahl der Open Source Algorithmen h�lt sich allerdings in Grenzen. Daher ist es relativ schwierig, die Konkurrenzalgorithmen fundiert zu analysieren. Ein guter Ansatz, um herauszufinden wie diese propriet�ren Algorithmen wirklich handeln, ist, sich die bekanntesten Einzelmethoden anzusehen, da h�chstwahrscheinlich ein Gro�teil der propriet�ren Algorithmen aus diesen Einzelmethoden unterschiedlichster Art aufgebaut sind. Wahrscheinlich ben�tzen sehr viele Konkurrenzprodukte die unterschiedlichsten Kombinationen aus \glspl{ma} und schm�cken diese mit einer Kombination aus den verschiedensten B�rsenindikatoren der technischen Analyse aus. Doch die Wenigsten betrachten den genauen Marktzustand in dem sie sich zur Zeit des Handelns befinden. Nat�rlich sind nur die simpelsten Marktzust�nde �ffentlich bekannt und zug�nglich und diese sind schon sehr abgen�tzt. Sie werden von einer so gro�en Anzahl an Softwareprodukten benutzt, dass man keinen oder nur mehr wenig Vorteil mehr daraus zieht, sie zu beachten. Deswegen liegt die einzige M�glichkeit in der Verbesserung von Software in diesem Punkt in der Entwicklung von eigenen Klassen von Marktzust�nden. W�hlt man hier richtig aus und investiert ausreichend Zeit in die Forschung, so kann man hiermit leicht an der Konkurrenz vorbeiziehen und performantere Algorithmen kreieren.

\newpage


\subsection{SOLL-Zustand}

Es gibt viele Algorithmen, allerdings sind wenige davon �ffentlich zug�nglich. Daher k�nnen auch �ber ihre Performance nur schlechte Aussagen getroffen werden. Man kann allerdings sagen, dass die auf dem Markt befindliche Software zur Testung von Algorithmen nicht sehr schnell und meist ohne Analyse arbeitet. Zus�tzlich verwenden viele Produkte eigene wenig bekannte Programmiersprachen, wie zum Beispiel \gls{mql5}, und arbeiten nicht mit einer verbreiteteren Sprache wie C, Java oder .net-Sprachen. Das Problem hierbei ist, dass sich viele Menschen keine neue Sprache aneignen wollen, sondern es bevorzugen, die ihnen bekannten Sprachen weiter zu ben�tzen. \\
\\
Mit einer speziell entwickelten \gls{bts} soll man nun einfacher und umfassender Algorithmen vergleichen k�nnen, ohne dabei auf eine neue Sprache umsteigen zu m�ssen. Mit einem solchen Vergleich wird es dem Projektteam m�glich sein, eine Reihe von Algorithmen auszuw�hlen, die bei verschiedenen Marktzust�nden unterschiedliche Performances gezeigt haben, und diese zu implementieren. Durch eine fr�he Erkennung der Performance soll man rechtzeitig, d.h. ohne Schaden, den Algorithmus wechseln k�nnen. Dazu ist auch notwendig, das Risiko des getesteten Algorithmus zu bestimmen und zu beschreiben. Zur Beschreibung ist eine Performancemessung (e.g. sharpe ratio) angedacht, die nach verschiedenen Marktzust�nden gegliedert werden soll. Eine Schwierigkeit ist dabei die Bestimmung der Zeitdauer eines Marktzustandes; als Beispiel: wie lange ein Seitw�rtstrend anh�lt und der Algorithmus damit rechnen soll.

\subsubsection{M�glichst geringes Risiko}

Durch die Entwicklung eines eigenen Algorithmus soll das Risiko, das ein Aktion�r tragen muss, m�glichst gering gehalten werden. Deshalb soll nun die M�glichkeit bestehen unterschiedliche Algorithmen bei verschiedenen Marktzust�nden zu vergleichen und daf�r mittels einer \gls{bts} eine Art Pr�fungsprotokoll anfertigen. Das sichert nicht nur den m�glichen Verkauf und die Funktionalit�t des Algorithmus, es erzeugt auch ein Vertrauen gegen�ber der Funktion, die sonst nur sehr vage zustande kommt.


\subsubsection{Weiterbildung der Projektteammitglieder}

Im Laufe des Projektes werden die Teamf�higkeit und die soziale Kompetenz der Mitglieder des Projektteams gest�rkt. Das Projektteam ist daran interessiert, Erfahrungen im Bereich der Finanzwirtschaft zu gewinnen. Mit der Durchf�hrung dieses Projekts wird den Mitgliedern das Konzept der B�rse und die damit verbundenen Algorithmen naher gebracht. Bei erfolgreicher Entwicklung eines performanten Algorithmus ist auch der Eigengebrauch f�r die Mitglieder des Projektteams nicht ausgeschlossen.%
%\section{Produktfunktionen} \label{section:produktfunktionen}

\subsection{Must-Have}

\begin{minipage}{\textwidth}
/F010/ \textbf{Vergangene Marktzust�nde bestimmen}

\begin{center}
\begin{tabular}{ | l | p{10cm} |}
\hline 
Beschreibung & Es sollen historische Marktzust�nde (innerhalb der letzten Jahre) auf transparenten
Aktienm�rkten, f�r die ein ausreichender Datenbestand vorhanden ist, automatisch bestimmt
werden. Sollten sich verschiedene gro�e M�rkte entgegen der Erwartung
unterschiedlich verhalten, sodass diese keiner einheitlichen Analyse unterzogen werden k�nnen,
soll prim�r der US-amerikanische Aktienmarkt untersucht werden. Hierbei handelt es sich um
eine Gruppierung von Zeitabschnitten nach gemeinsamen Kriterien.\\  \hline
Aufwand & Hoch \\ \hline
Nutzen & Hoch \\ \hline
Ziel & Ermittlung von Klassen f�r Marktzust�nde. \\ \hline
Vorbedingungen & - \\ \hline
Nachbedingungen & - \\ \hline
\end{tabular}
\end{center}
\end{minipage}\\[4ex] 

\begin{minipage}{\textwidth}
/F020/ \textbf{Aktuellen Marktzustand bestimmen}

\begin{center}
\begin{tabular}{ | l | p{10cm} |}
\hline 
Beschreibung & Dabei soll darauf geachtet werden, dass f�r eine fr�he Erkennung m�glicherweise nur ein Teil
der Daten vorhanden ist, die f�r die historische Analyse herangezogen werden k�nnen.\\  \hline
Aufwand & Mittel \\ \hline
Nutzen & Hoch \\ \hline
Ziel & Zuordnung des aktuellen Marktzustandes zu einem bereits bekannten. \\ \hline
Vorbedingungen & /F010/ Vergangene Marktzust�nde bestimmen \\ \hline
Nachbedingungen & - \\ \hline
\end{tabular}
\end{center}
\end{minipage}\\[4ex] 

\begin{minipage}{\textwidth}
/F110/ \textbf{Trends erkennen}

\begin{center}
\begin{tabular}{ | l | p{10cm} |}
\hline 
Beschreibung & Durch \glspl{ma} soll es m�glich sein, Trends in Aktienkursen zu identifizieren. Dazu
kommen verschiedene Crossover-Verfahren (double- / triple-crossover) oder Indikatoren, wie
der MACD (Moving Average Convergence Divergence) in Frage. Es soll eine statistisch
m�glichst profitable Variante hierf�r gefunden werden, die aufscheinende nachhaltige Trends
m�glichst gut erkennt.\\  \hline
Aufwand & Hoch \\ \hline
Nutzen & Hoch \\ \hline
Ziel & Fr�hzeitige m�glichst profitable Erkennung von Trends. \\ \hline
Vorbedingungen & - \\ \hline
Nachbedingungen & - \\ \hline
\end{tabular}
\end{center}
\end{minipage}\\[4ex] 

\begin{minipage}{\textwidth}
/F120/ \textbf{\gls{ma}-Dauer bestimmen}

\begin{center}
\begin{tabular}{ | l | p{10cm} |}
\hline 
Beschreibung & Je nachdem, wie lange ein Trend andauert, bedingt eine Trenderkennung andere \gls{ma}(-Paare)
mit unterschiedlichen Laufzeiten. Durch Backtesting sollen viele verschiedene Varianten
automatisch getestet werden k�nnen, um den statistisch besten Parametersatz zu ermitteln.\\  \hline
Aufwand & Niedrig \\ \hline
Nutzen & Mittel \\ \hline
Ziel & Erarbeitung eines optimalen Parametersatzes f�r ein \gls{ma}-Paar. \\ \hline
Vorbedingungen & - \\ \hline
Nachbedingungen & - \\ \hline
\end{tabular}
\end{center}
\end{minipage}\\[4ex] 

\begin{minipage}{\textwidth}
/F130/ \textbf{An Marktzustand anpassen}

\begin{center}
\begin{tabular}{ | l | p{10cm} |}
\hline 
Beschreibung & Der Algorithmus soll sich durch Parameterver�nderung an den erkannten Marktzustand zur
Optimierung der Performance anpassen. Dies kann beispielsweise durch Ver�ndern der \gls{ma}-Paare
oder durch Anpassung der Market Exposure und damit des Risikos erfolgen.
Dazu \textit{k�nnen} die Implikationen durch Nachforschung bekannt sein, woraufhin ein Modell
angewandt wird, m�ssen aber nicht, da auch induktiv aus den Implikationen gelernt werden
kann, wonach automatisch ein Modell entsteht (\textit{Maschinelles Lernen}). Dabei werden f�r die
unterschiedlichen Markzust�nde verschiedene Parameters�tze durchprobiert.\\  \hline
Aufwand & Hoch \\ \hline
Nutzen & Hoch \\ \hline
Ziel & Anpassung der Hauptfunktionen des Algorithmus an den aktuellen Marktzustand. \\ \hline
Vorbedingungen & /F020/ Aktuellen Marktzustand bestimmen \\ \hline
Nachbedingungen & - \\ \hline
\end{tabular}
\end{center}
\end{minipage}\\[4ex] 

\begin{minipage}{\textwidth}
/F140/ \textbf{Signale generieren}

\begin{center}
\begin{tabular}{ | l | p{10cm} |}
\hline 
Beschreibung & Signalgeben bei potentiellen Einstiegspunkten (long signal) und Ausstiegspunkten (short
signal).\\  \hline
Aufwand & Niedrig \\ \hline
Nutzen & Hoch \\ \hline
Ziel & R�ckgabe von Handelssignalen. \\ \hline
Vorbedingungen & /F110/ Trends erkennen, /F130/ An Marktzustand anpassen, /F120/ \gls{ma}-Dauer bestimmen \\ \hline
Nachbedingungen & /F160/ Signale filtern \\ \hline
\end{tabular}
\end{center}
\end{minipage}\\[4ex] 

\begin{minipage}{\textwidth}
/F150/ \textbf{Trend-Nachhaltigkeit bestimmen}

\begin{center}
\begin{tabular}{ | l | p{10cm} |}
\hline 
Beschreibung & Durch geeignete Support- und Resistance-Indikatoren soll die Nachhaltigkeit eines Trends
bestimmt werden (beispielsweise \gls{rsi}, \gls{cci} oder \glspl{ma}), um den Ausstiegspunkt zu
optimieren.\\  \hline
Aufwand & Mittel \\ \hline
Nutzen & Hoch \\ \hline
Ziel & Festellen der Nachhaltigkeit erkannter Trends. \\ \hline
Vorbedingungen & /F140/ Signale generieren \\ \hline
Nachbedingungen & - \\ \hline
\end{tabular}
\end{center}
\end{minipage}\\[4ex] 

\begin{minipage}{\textwidth}
/F160/ \textbf{Signale filtern}

\begin{center}
\begin{tabular}{ | l | p{10cm} |}
\hline 
Beschreibung & Zur Verminderung von unprofitablen, zu kurzen Trades sollen insbesondere Kaufsignale
gefiltert werden. Die Trenderkennung k�nnte des �fteren zu kurz anhaltende Trends
erkennen, indem beispielsweise ein MA-Crossover nur f�r kurze Zeit besteht. Durch das
Einf�hren eines Schwellenwertes (threshold), der �berschritten werden muss, oder einer
bestimmten Zeitspanne, die ein Signal �berdauern muss, k�nnen zu kurze Trades vermindert
werden, wenn sich im Backtesting dadurch ein Vorteil herausgestellt hat.\\  \hline
Aufwand & Mittel \\ \hline
Nutzen & Hoch \\ \hline
Ziel & Filterung der unbrauchbaren Signale. \\ \hline
Vorbedingungen & /F140/ Signale generieren, /F150/ Trend-Nachhaltigkeit bestimmen \\ \hline
Nachbedingungen & - \\ \hline
\end{tabular}
\end{center}
\end{minipage}\\[4ex] 

\begin{minipage}{\textwidth}
/F210/ \textbf{Performance berechnen}

\begin{center}
\begin{tabular}{ | l | p{10cm} |}
\hline 
Beschreibung & Die relative Performance eines Algorithmus soll in Prozent der Kapitalver�nderung berechnet
werden.\\  \hline
Aufwand & Mittel \\ \hline
Nutzen & Hoch \\ \hline
Ziel & Berechnung der relativen Performance eines Algorithmus. \\ \hline
Vorbedingungen & - \\ \hline
Nachbedingungen & - \\ \hline
\end{tabular}
\end{center}
\end{minipage}\\[4ex] 

\begin{minipage}{\textwidth}
/F220/ \textbf{Gewinn/Risiko-Verh�ltnis berechnen}

\begin{center}
\begin{tabular}{ | l | p{10cm} |}
\hline 
Beschreibung & Bestimmung des Risikos des Algorithmus (beispielsweise anhand der Volatilit�t) in
Verbindung mit der Performance (e.g. sharpe ratio).\\  \hline
Aufwand & Niedrig \\ \hline
Nutzen & Mittel \\ \hline
Ziel & Bestimmung des Gewinn/Risiko-Verh�ltnisses eines Algorithmus. \\ \hline
Vorbedingungen & /F210/ Performance berechnen \\ \hline
Nachbedingungen & - \\ \hline
\end{tabular}
\end{center}
\end{minipage}\\[4ex] %
%\section{Technische Machbarkeit}\label{section:technischemachbarkeit}
\subsection{Variantenbildung}
\subsubsection{Programmiersprachen}
Die \gls{bts} kann man in jeder erdenklichen Programmiersprache schreiben, allerdings ist es wichtig daran zu denken, dass das Programm einerseits effizient arbeiten soll und deswegen hardwarenahe rechnet, und andererseits hat das Projektteam mit manchen Programmiersprachen keinerlei Erfahrung.\\
Die allgemeine Funktionalit�t muss das lesen und schreiben von Dateien sein, aber auch das algorithmische Rechnen soll effizient funktionieren. F�r das Team kommen daher 3 M�glichkeiten in Frage: Eine L�sung in reinem C++, welches sehr hardwarenahe arbeitet, eine Mischung aus F\# und C\#, mit der eine parallelisierte Berechnung m�glich w�re, und eine Java-L�sung, bei der das Team die gr��te Erfahrung mitbringt. \\
Bei der Kombination agiert C\# als Handlungs- und Steuerkern und F\# als funktionale Programmiersprache, als Rechenkern und \"Mastermind\" der Applikation, welches die Entscheidungen trifft. Hierbei wird einerseits eine enorm hohe Arbeitsgeschwindigkeit erm�glicht, da die beiden Sprachen relativ hardwarenah agieren und andererseits besteht der nicht zu untersch�tzende Vorteil bzw. die M�glichkeit, den Rechenkern auf ein externes System outzusourcen, welches zum Beispiel enorme Rechenkapazit�ten aufweisen k�nnte und somit viel komplexere und effizientere Algorithmen in annehmbarer Zeit durchrechnen und abh�ngig davon mehr gewinnbringende Entscheidungen treffen k�nnte. Dabei sollte es auch bei sp�teren Erweiterungen des Programms zu keinem signifikanten Geschwindigkeitsabfall kommen.


\begin{center}

\begin{tabular}{ | c | p{2.6cm} | p{1.7cm} | p{0.5cm} |p{0.5cm}|p{0.5cm}|p{0.5cm}|p{0.7cm}|p{0.7cm}|}
\hline 
\multicolumn{2}{|p{1.5cm}|}{ }  & Gewicht\-ung & \multicolumn{2}{p{1.5cm}|}{\textbf{C++ R*G}} & \multicolumn{2}{p{1.5cm}|}{\textbf{Java R*G}} & \multicolumn{2}{|p{1.5cm}|}{\textbf{C\#F\# R*G}}\\ \hline
\multirow{6}{*}{Einfachheit} & Aufwand Coding & 10\% & 3 & 30 & 1 & 10 & 2 & 20 \\ \cline{2-9}
& Bedienung/ Wartung & 6\% &3&9&2&6&1&3\\ \cline{2-9}
& Update &3\%&3&9&2&6&1&3\\ \cline{2-9}
& Integration &5\%&3&15&2&10&1&5\\  \cline{2-9}
& Kenntnisse &6\%&3&18&1&6&2&12\\ \cline{2-9}
& \textbf{Gesamt}&30\%&3&90&2&44&1&46\\ \hline
\multirow{5}{*}{Leistung}& �bertragungs-zeit &6\%&1&6&3&18&2&12\\ \cline{2-9}
& Absturz\-sicherheit &5\%&1&5&2&10&3&15\\ \cline{2-9}
& Ressourcen-verbrauch &3\%&1&3&3&9&2&6\\ \cline{2-9}
& Datenumfang &1\%&1&1&3&3&2&2\\ \cline{2-9}
& \textbf{Gesamt} &15\%&1&15&3&50&2&35\\ \hline
\multirow{5}{*}{Kosten}& Lizenzen &10\%&1&10&1&10&1&10\\ \cline{2-9}
& Support &5\%&3&15&1&5&2&10\\ \cline{2-9}
& Betriebs-kosten &5\%&1&5&1&5&1&5\\ \cline{2-9}
& Dokumen\-tation &5\%&1&5&2&20&3&15\\ \cline{2-9}
& \textbf{Gesamt} &15\%&1&30&1&40&2&40\\ \hline
\multirow{4}{*}{Dokumentation}& Verf�gbarkeit &10\%&3&30&2&20&1&10\\ \cline{2-9}
& Voll\-st�ndigkeit &10\%&3&30&2&20&1&10\\ \cline{2-9}
& Qualit�t &10\%&2&20&1&10&1&10\\ \cline{2-9}
& \textbf{Gesamt} &30\%&3&80&2&50&1&30\\ \hline
\end{tabular}

\end{center} 

\begin{center}

\begin{tabular}{|l|r|p{0.8cm}|p{0.8cm}|p{0.8cm}|p{0.8cm}|p{0.8cm}|p{0.8cm}|} \hline
Kapitel&Gewichtung&\multicolumn{2}{p{1.8cm}|}{\textbf{C++}}&\multicolumn{2}{p{1.8cm}|}{\textbf{Java}}&\multicolumn{2}{p{1.8cm}|}{\textbf{C\#/F\#}}\\ \hline
Einfachheit&30\%&3&90&2&44&1&46 \\ \hline
Leistung&15\%&1&15&3&50&2&35 \\ \hline
Kosten&15\%&2&30&2&40&1&40 \\ \hline
Dokumentation&30\%&3&80&2&50&1&30 \\ \hline
\end{tabular}

\end{center}

\begin{center}
\begin{tabular}{|l|l|l|l|} \hline
Gesamtbewertung&&&\\ \hline
Endreihung &3&2&1\\ \hline
\end{tabular}
\end{center}

Aus der Nutzwertanalyse kann man entnehmen, dass die C\#/F\# Kombination als die favorisierte M�glichkeit ausgeht, weitere Vorteile die sich aus der Wahl dieser Mischung ergeben sind: gute Kenntnisse der Programmiersprachen, tolle Community und die Einfachheit, sowie die Erweiterbarkeit. Bei dieser L�sung wird die Steuereinheit vom C\# Teil des Programms �bernommen und die Rechenaufgaben werden von dem F\# Teil bearbeitet. Au�erdem ist das .net-Framework sehr beliebt, deswegen kann man damit rechnen das bei einem Problem gen�gend Helfer gefunden werden k�nnen.

\subsubsection{\gls{bts}}
Bei der \gls{bts} handelt es sich wie bereits erwaehnt, um eine Software, die Algorithmen auf ihre Performance und weitere wichtige Kriterien testet.
Um diese Aufgabe zu l�sen, m�ssen zwei wichtige Entscheidungen zur Art der Realisierung getroffen werden:

\begin{itemize}
	\item Ist es sinnvoller diese Software �ber eine \gls{gui} steuern zu k�nnen, oder reicht es wenn sie in der Konsole arbeitet? 
	\item Auf welche Art und weise soll der Algorithmus der getestet werden soll, zur Laufzeit in die Software eingebunden werden.?
\end{itemize}

Gehen wir nun also zuerst der Frage nach dem User-Interface nach.\\
Bei einem gro�en Anteil der Zielgruppe dieser \gls{bts} handelt es sich um Chartisten oder andere sehr visuelle Personen. Diese w�rden vermutlich eine graphische Oberfl�che bevorzugen, da sie dabei z.B. auch ihren Algorithmus so erweitern k�nnten, dass dieser die Charts die ihm zu Grunde liegen in einem neuen Fenster darstellt und man dadurch einen noch besseren �berblick �ber das Geschehen des Algorithmus erlangen k�nnte. Andererseits wird es sich aber auch bei einem gro�en Teil der Zielgruppe, um Technische Programmierer und Mathematiker handeln, die wieder nicht so viel Wert auf eine graphische Oberfl�che legen, da sie eher in Zahlen, als in Bildern denken. Allerdings w�re mit Sicherheit fast niemand dieser Gruppe strikt gegen eine Graphische Oberfl�che. Es ist hierbei nur abzuwiegen, ob der erh�hte Aufwand den die Codierung einer solchen \gls{gui} mit sich bringen w�rde, sinnvoll ist und wirklich eine h�here Anzahl an Kunden interessieren w�rde. Mit einer \gls{gui} w�re es allerdings auch erheblich einfacher, einen Algorithmus einzubinden.\\
\\
Es erscheint dem Projektteam aufgrund der Programmiersprachenwahl am sinnvollsten, dass die User den Algorithmus in der Sprache F\# schreiben und diesen als \gls{dll} konvertieren m�ssen, um ihn in die Software zu integrieren. Grunds�tzlich ist es dadurch in jedem Fall notwendig, dass mit der \gls{bts} als Download zus�tzlich ein Interface(zur Beschreibung von Methoden-Namen die benutzt werden, usw.) und eine Beispiel-\gls{dll} bereitgestellt werden, damit die Nutzer der Software einen schnellen und einfachen �berblick bekommen, wie sie ihre \glspl{dll} aufbereiten m�ssen, um diese problemlos die \gls{bts} integrieren zu k�nnen. Durch die Frage, wie die Integration eine solchen \gls{dll} nun wirklich am einfachsten f�r unsere zuk�nftigen Benutzer w�re, ergeben sich zwei M�glichkeiten zur Aufbereitung und Einbindung eines Algorithmus. Man k�nnte den Algorithmus einfach zur Laufzeit in die Software laden, indem z.B. die Software ein Dateisystem-Browsing anbietet, indem man die Algorithmus-\gls{dll} einfach ausw�hlt und diese wird automatisch in die Software integriert und so der Algorithmus getestet. W�rde die Software als Konsolenapplikation laufen, w�re dies schon etwas komplizierter f�r unerfahrene Nutzer, da man ein bestimmtes Kommando und den direkten Pfad zur Algorithmus-\gls{dll} selbst eintippen m�sste. Eine ganz andere M�glichkeit w�re es den Algorithmus als eigenst�ndige Applikation bereitzustellen, auf die man �ber Sockets eine Verbindung aufbauen und die Rechenoperationen an sie auslagern kann. Auf diese Art und Weise w�re es ebenfalls m�glich den Algorithmus zu testen, allerdings m�sste das Projektteam eine viel Umfangreichere Beispiel-\gls{dll} bereitstellen, in der die Eigenst�ndigkeit der Software, sowie die Erreichbarkeit �ber Sockets direkt nach ihrem Start bereits vordefiniert sein m�ssten. Dies w�rde zu einem etwas erh�hten Aufwand f�r das Projektteam f�hren, wobei dies nicht das Hauptproblem dieser Variante darstellt. Denn durch die umfangreiche anfangs mit Sicherheit un�bersichtliche Beispiel-\gls{dll} w�rden programmiertechnisch unerfahrene Benutzer schnell zur�ckschrecken und lieber Konkurrenzprodukte benutzen, als sich in dieses Produkt aufwendig einzulesen.\\
\\
Durch diesen gesamten Analysevorgang kam das Projektteam zu dem Schluss, dass es f�r die zuk�nftige Zielgruppe des Produkts wohl am einfachsten und damit auch am sinnvollsten w�re, dieses Produkt mit einer \gls{gui} zu realisieren, da es sich bei der Zielgruppe meist zwar um B�rsenerfahrene Personen handelt, diese aber wahrscheinlich nicht so standfest in der Welt des .net-Frameworks sind. Au�erdem erm�glicht eine simple \gls{gui}, die Einbindung eines Algorithmus mittels Dateosystem-Browsing, ohne dass der Benutzer m�hsamst den Pfad selbst suchen und in die Software einf�gen muss. Als Variante zur Einbindung es algorithmus erscheint es dem Projektteam ebenfalls am sinnvollsten den Algorithmus einfach �ber ein Interface mit einer ganz normalen \gls{dll} einzubinden, da es aufwendiger und wahrscheinlich auch langsamer w�re, den Algrithmus als eingenst�ndige Software �ber Sockets anzustprechen.%
%% Wirtschaftliche Machbarkeit

\section{Wirtschaftliche Machbarkeit} \label{section:wirtschaftlichemachbarkeit}

\subsection{Aufwandsabsch�tzung}

\subsubsection{Personalaufwand}

\begin{figure}[h]
	\centering
		\includegraphics[width=0.80\textwidth]{graphics/wirmachbarkeit/FunctionPointsAnalysis.JPG}
	\caption{Functions-Point-Analyse zu Noctua.}
	\label{fig:FunctionPointsAnalysis}
\end{figure}

\begin{figure}
	\centering
		\includegraphics[width=0.80\textwidth]{graphics/wirmachbarkeit/IBM_FPS_Analysis.png}
	\caption{IBM Umrechnung von Functions-Points in Personenmonate aufgrund von Erfahrungswerten.}
	\label{fig:IBM_FPS_Analysis}
\end{figure}


Aus der Abbildung \ref{fig:FunctionPointsAnalysis} ist ersichtlich, dass die Function-Points-Analyse ca. 90 Punkte ergibt.
Anhand der Tabelle aus Abbildung \ref{fig:IBM_FPS_Analysis} ergibt sich f�r 90 Function-Points eine ungef�hre Projektdauer von 3 Personenmonaten.
Dies entspricht bei einer Umrechnung von 40 Stunden pro Woche eine Gesamtstundenanzahl von 480 Arbeitsstunden. Dies wiederum geteilt durch die
Anzahl der Projektteammitglieder ergibt das eine Workload von ca. 160 Stunden pro Projektteammigtlied.\\
\\
Bei einem Stundensatz von \EUR{50,00} pro Arbeitsstunde ergibt dies ein ben�tigtes Kapital von \EUR{24.000,00} f�r das gesamte Projekt Noctua. 

\subsubsection{Materialaufwand}

Da es sich bei diesem Projekt um ein reines Softwareprojekt handelt, bezieht sich der Materialaufwand ausschlie�lich auf die Lizenzkosten f�r die verwendete Software. Verbrauchsmaterial kommt in diesem Projekt nicht zum Einsatz.\\
\\
Die Lizenzenkosten belaufen sich auf folgendes:
\begin{itemize}
	\item 3 * \EUR{1.528,00} f�r eine Microsoft Visual Studio Professional Lizenz f�r jeden unserer Computer. Diese dient als Entwicklungsumgebung.
	\item 1 * ca. \EUR{191,00} f�r eine Microsoft Visio Standard 2010, zur Modellierung von Diagrammen.
	\item 3 * ca. \EUR{92,00} f�r eine Windows 7 Home Premium Lizenz f�r jeden unserer Computer.
	\item 1 * ca. \EUR{345,00} f�r eine Microsoft Project Standard 2010 Lizenz, zur Planung des Projekts.
	\item 1 * ca. \EUR{65,00} f�r eine 2-9 Personen Lizenz von SmartGit mit 1 Jahr Support. Dies wird als \gls{git}-Client benutzt.
\end{itemize}
Dies ergibt einen Gesamtmaterialaufwand von ca. \EUR{5.461,00}.
\subsubsection{Investitionskapital}

Der aller wichtigste Teil des Investitionskapitals ist ein Laptop f�r jedes der Projektteammitglieder, um die Programmierarbeit �berhaupt zu erm�glichen.
Die Abschreibung daf�r bel�uft sich laut der \gls{afa} f�r Laptops ab \EUR{400,00} auf ca. \EUR{1.500,00}.\\
\\
Zus�tzlich fallen kosten f�r den Arbeitsraum an. Man kann f�r drei Personen von einem 30 Quadratmeter gro�en B�roraum inklusive WC-Einrichtungen ausgehen.
Inklusive Heizung fallen hierf�r ca. \EUR{13,00}/m$^2$/Monat an. Das bel�uft sich dann auf ca. \EUR{4.680,00}/Jahr. Zus�tzlich kann man ca. \EUR{500,00}/Jahr
f�r Strom und Wasser annehmen.\\
\\
Dadurch bel�uft sich das gesamte ben�tigte Investitionskapital auf ca. \EUR{6.680,00} f�r die gesamte Projektdauer.

\subsection{Nutzen}

Der Algorithmus kann in bestehende Tradingsoftware intergriert werden, um dadurch die Entscheidungen in Form von Orders �ber einen Online-Broker an die B�rse weiterzuleiten.\\
	Angenommen es wird mit dem Algorithmus ein Fond mit einem angestellten Trader gemanged. Der Trader erh�lt ein Bruttogehalt von \EUR{3.200} im Monat und eine Provision von 10\% des erwirtschafteten Gewinns. Daher ist festzustellen, ab welcher Fondgr��e es sich auszahlt diesen Trader einzustellen und ab welcher Gr��e zus�tzlich ein Programmierer wirtschaftlich ist, um den Algorithmus weiter zu entwickeln.

\begin{figure}[h]
	\centering
		\includegraphics[width=0.80\textwidth]{graphics/wirmachbarkeit/maxima_nutzen.png}
	\caption[Break Even Point Analyse]{Break Even Point Analyse bei 8\% Gewinn mit Trader und zus�tzlichem Programmierer}
	\label{fig:maxima_nutzen}
\end{figure}

In der Grafik \ref{fig:maxima_nutzen} ist ersichtlich, dass der Fonds f�r einen profitablen Betrieb bei einem durchschnittlichen j�hrlichen Gewinn von \emph{8\%} mindestens ein Volumen von \EUR{1.185.185,2} verwalten muss, um einen Gewinn zu erwirtschaften. Wenn zus�tzlich noch ein Programmierer zur Weiterentwicklung angestellt werden soll muss das Volumen auf \EUR{2.296.296,33} steigen. Dabei werden \emph{20\%} des Gewinns als Provision berechnet und zus�tzlich noch \emph{2\%} gewinnunabh�ngig.

\subsection{Pr�fen der Risiken}
\subsubsection{Personenausfall}
Eintrittswahrscheinlichkeit:  gering \\
Auswirkungen:  								hoch   \\

In dem unerwarteten Fall, dass ein Teammitglied l�ngerfristig ausf�llt, muss es m�glich sein die Arbeitsaufgaben dementsprechend neu aufteilen zu k�nnen.
Folgende F�lle k�nnten auftreten:
\begin{itemize}
	\item Streit im Team
	\item Ausfall durch Krankheit oder Tod eines Teammitglieds
	\item Austritt eines Teammitglieds aus dem Projekt
	\item Der Auftraggeber k�nnte aufgrund von Unklarheiten den Projektabbruch initiieren 
	\item Es kann passieren, dass von Seite des Auftragsgebers pl�tzlich kein Interesse an der Umsetzung des Produktes mehr gegeben ist, und es dadurch zu extremem Zeitverzug kommt, was bis zum Abbruch f�hren kann
	\item Interessensverlusst eines Teammitglieds, und damit verbundene minderwertigere Arbeit 
\end{itemize}

Folgende pr�ventive Ma�nahmen werden eingef�hrt:
\begin{itemize}
	\item Regeln f�r den Umgang innerhalb des Projekts
	\item Ausreichendes Interesse jedes Mitglieds und keine leistungstechnische Probleme 
	\item Gutes Verh�ltnis mit den Auftraggebern
\end{itemize}

\subsubsection{Zeitliche Risiken}

Eintrittswahrscheinlichkeit:  gering \\
Auswirkungen:  								mittel \\

Die Aufwands- und Zeitsch�tzung basiert auf dem derzeitigen Lastenheft des Auftraggebers und stellt eine zeitgerechte Fertigstellung sicher. Sollten sich jedoch die Anforderungen des Kunden w�hrend des Projekts �ndern, so wird sich das mit gro�er Wahrscheinlichkeit verz�gernd auf den Fertigstellungstermin auswirken. Die mit dem Kunden vereinbarte Funktionsanalyse und die Meilensteine mit gemeinsam festgelegten Qualit�tskriterien sollten jedoch diesem Risiko entgegenwirken.

\subsubsection{Technische Risiken}
\label{subsection:Technische Risiken}
\textbf{Datenverlust}\\
Eintrittswahrscheinlichkeit:  gering \\
Auswirkungen:  								mittel \\

Aufgrund der nicht auszuschlie�enden Gefahr des Datenverlusts, muss daf�r gesorgt werden die Sicherheit der Daten, sowie auch die Verf�gbarkeit dieser zu garantieren. Dieses Problem wird mithilfe eines \gls{git} gel�st, durch diesen Server ist es m�glich die Versionen der Software immer zug�nglich zu machen und zus�tzlich die Daten auf den Computern der Projektmitgliedern zu speichern. \\ \\

\textbf{Hardwareausfall} \\
Eintrittswahrscheinlichkeit: gering \\
Auswirkung: mittel \\

Es kann ohne jede Vorwarnung immer und �berall ein Hardwareausfall passieren, dies kann selbst verschuldet, aber auch pl�tzlich passieren. Um mit einem solchen Ausfall zurecht zu kommen besitzt das Projektteam einen Ersatzlaptop um das gezielte Arbeiten auch nach dem 
Ausfall garantieren zu k�nnen. \\ \\

\textbf{Versionsverlust} \\
Eintrittswahrscheinlichkeit: sehr gering \\
Auswirkung: mittel \\

Bei den Versuchen mit dem Algorithmus wird andauernd etwas in der Datei ver�ndert. Bei dieser T�tigkeit kann es passieren das die Zusammenh�nge zwischen 1 oder mehreren Versionen des Algorithmus verloren gehen. Bei so einem Verlust kann auch das grundliegende Verst�ndnis f�r die jeweilige Version verloren gehen.\\
Folgende pr�ventive Ma�nahmen werden eingef�hrt:
\begin{itemize}
	\item Verwendung eines \gls{git}, f�r die automatische Versionierung
	\item Zwingende Benutzung der Bugtrackingsoftware
	\item Kommunikation im Team �ber die �nderungen am Algorithmus
\end{itemize}%
% Chapter3

\chapter{The Proposed Architecture} \label{chapter:architecture}

\begin{quotation}
``The market is not an invention of capitalism. It has existed for centuries. It is an invention of civilization.``
\begin{flushright}
(Mikhail Gorbachev)
\end{flushright}
\end{quotation}



\section{Introduction}
%
% Chapter4
\chapter{Implementation} \label{chapter:thevetestcase}
%
% Chapter5

\chapter{Produktdaten} \label{chapter:Produktdaten}

%
\chapter{Produktleistungen} \label{chapter:Produktleistungen}

%


\addcontentsline{toc}{chapter}{Glossary} 
\printglossary[type=\acronymtype]
\glsaddall

\printglossary[type=\acronymtype]
%\printglossary[type=\acronymtype,style=listwithwidth]

%% include appendix
\begin{appendix}
\chapter{Appendix\label{appendix_A}}
%
\end{appendix}

% Use of the sorted IEEE style, with changes:
% "dashification" was disabled


\cleardoublepage

% IEEE Style
\bibliographystyle{sty/IEEEtranS}

% GATHER
%\input "literatur/bib.bib"
\phantomsection{}
\addcontentsline{toc}{chapter}{\bibname}
\pagestyle{myheadings}\markboth{\bibname}{\bibname}
%\bibliography{bib}

%%% generate index
%\clearpage%
%\markboth{\indexname}{\indexname}%
%\printindex%
%\addcontentsline{toc}{chapter}{\numberline{}\indexname}%

%% include affidavit
\thispagestyle{empty}
\vspace*{2cm}
\begin{center}
{\bf \sf \huge Erkl{\"a}rung}
\end{center}
{\sf \vspace{1cm} Hiermit erkl{\"a}ren wir, dass die vorliegende
Arbeit ohne unzul{\"a}ssige Hilfe Dritter und ohne Benutzung
anderer als der angegebenen Hilfsmittel angefertigt wurde. Die aus
anderen Quellen oder indirekt �bernommenen Daten und Konzepte sind
unter Angabe der Quelle gekennzeichnet.

Die Arbeit wurde bisher weder im In- noch im Ausland in gleicher
oder in {\"a}hnlicher Form in anderen Pr{\"u}fungsverfahren
vorgelegt.
\\[1.5cm]
Wien, im \monthdis
\\[2cm]
Name1
\\[2cm]
Name2
\\[2cm]
Name3
\\[2cm]
Name4
\\[2cm]
}%end sf
%



\end{document}
%%%%%%%%%%%%%%%%%%%%%%%%%%%%%%%%%%%%%%%%%%%%%%%%%%%%%%%%%%%%%%%%%%%%%%%%%%%%%%%%%%%%%%%%%%%%%%%%%%%
%%%%%%%%%%%%%%%%%%%%%%%%%%%%%%%%%%%%%%%%%% End Dokument %%%%%%%%%%%%%%%%%%%%%%%%%%%%%%%%%%%%%%%%%%%
%%%%%%%%%%%%%%%%%%%%%%%%%%%%%%%%%%%%%%%%%%%%%%%%%%%%%%%%%%%%%%%%%%%%%%%%%%%%%%%%%%%%%%%%%%%%%%%%%%%
