\section{Technische Machbarkeit}\label{section:Technische Machbarkeit}
\subsection{Variantenbildung}
Gr��tenteils gibt es derzeit f�r Programme dieser Art marktweit einen �blichen L�sungsansatz:
Hierbei wird in C++ sowohl die Konsolenapplikation als auch die graphische Benutzeroberfl�che entworfen, gecodet und schlussendlich implementiert. Dies erm�glicht einerseits ein sehr zentrales Arbeiten, da alles am selben Rechner erfolgt. Andererseits muss man jedoch oft auf C-Befehle zur�ckgreifen, um auf eine annehmbare Arbeitsgeschwindigkeit zu kommen.
Eine weitere M�glichkeit um die Applikation zu realisieren, ist ein �hnlicher Aufbau wie der in der C++-Variante. Das Ganze wird hierbei jedoch in Java umgesetzt. Das hei�t Konsole und GUI werden in Java gecodet und erm�glichen somit vollkommen plattformunabh�ngiges Arbeiten. Jedoch gibt es hier erhebliche Probleme mit der Geschwindigkeit, da Java �ber die JVM arbeitet und somit relativ hardwarefern agiert.
Die dritte M�glichkeit w�re, das Ganze �ber ein Programmiersprachen-\"Comboteam\" von C\# und F\# umzusetzen. Hierbei agiert C\# als Handlungs- und Steuerkern und F\# als funktionale Programmiersprache, als Rechenkern und \"Mastermind\" der Applikation, welches die Entscheidungen trifft. Hierbei wird einerseits eine enorm hohe Arbeitsgeschwindigkeit erm�glicht, da die beiden Sprachen relativ hardwarenah agieren und andererseits besteht der nicht zu untersch�tzende Vorteil bzw. die M�glichkeit, den Rechenkern auf ein externes System outzusourcen, welches zum Beispiel enorme Rechenkapazit�ten aufweisen k�nnte und somit viel komplexere und effizientere Algorithmen in annehmbarer Zeit durchrechnen und abh�ngig davon mehr gewinnbringende Entscheidungen treffen k�nnte. Dabei sollte es auch bei sp�teren Erweiterungen des Programms zu keinem signifikanten Geschwindigkeitsabfall kommen.


\begin{center}

\begin{tabular}{ | c | p{2.6cm} | c | p{0.5cm} |p{0.5cm}|p{0.5cm}|p{0.5cm}|p{0.5cm}|p{0.5cm}|}
\hline 
�& & Gewicht & \multicolumn{2}{p{1.5cm}|}{\textbf{C++ R*G}} & \multicolumn{2}{p{1.5cm}|}{\textbf{Java R*G}} & \multicolumn{2}{p{1.5cm}|}{\textbf{C/F R*G}}\\ \hline
\multirow{6}{*}{Einfachheit} & Aufwand Coding & 10\% & 3 & 30 & 1 & 10 & 2 & 20 \\ \cline{2-9}
& Bedinung/ Wartung & 6\% &3&9&2&6&1&3\\ \cline{2-9}
& Update &3\%&3&9&2&6&1&3\\ \cline{2-9}
& Integration &5\%&3&15&2&10&1&5\\  \cline{2-9}
& Kenntnisse &6\%&3&18&1&6&2&12\\ \cline{2-9}
& \textbf{Gesamt}&30\%&3&90&2&44&1&46\\ \hline
\multirow{5}{*}{Leistung}& �bertragungs-zeit &6\%&1&6&3&18&2&12\\ \cline{2-9}
& Absturz-sicherheit &5\%&1&5&2&10&3&15\\ \cline{2-9}
& Ressourcen-verbrauch &3\%&1&3&3&9&2&6\\ \cline{2-9}
& Datenumfang &1\%&1&1&3&3&2&2\\ \cline{2-9}
& \textbf{Gesamt} &15\%&1&15&3&50&2&35\\ \hline
\multirow{5}{*}{Kosten}& Lizenzen &10\%&1&10&1&10&&\\ \cline{2-9}
& Support &5\%&3&15&1&5&2&10\\ \cline{2-9}
& Betriebs-kosten &\%&&&&&&\\ \cline{2-9}
& Dokumen-tation &5\%&1&5&2&20&3&15\\ \cline{2-9}
& \textbf{Gesamt} &\%&&&&&&\\ \hline
\multirow{4}{*}{Dokumentation}& Verf�gbarkeit &10\%&3&30&2&20&1&10\\ \cline{2-9}
& Voll-st�ndigkeit &10\%&3&30&2&20&1&10\\ \cline{2-9}
& Qualit�t &10\%&2&20&1&10&1&10\\ \cline{2-9}
& \textbf{Gesamt} &30\%&3&80&2&50&1&30\\ \hline
\end{tabular}

\end{center} 