% !TeX root = ../../Noctua_Diplomarbeit.tex

\section{Management Summary} \label{section:management_summary}

Die Machbarkeitsstudie des Projektes NOCTUA behandelte die kritischen Punkte der Algorithmuskonzeption und -entwicklung und der \gls{bts}.\\
	Ein wichtiger Bestandteil waren Nachforschungen im Bereich der finanzwirtschaftlichen Grundlagen und der technischen Analyse. Dabei wurden einige \linebreak Trend\-folge\-mecha\-nismen und Oszillatoren beschrieben, um aus diesen und �hnlichen Mechanismen im Laufe des Projektes einen eigenen Algorithmus entwickeln zu k�nnen.\\
\\
Da es wichtig ist, einen Algorithmus auch w�hrend der Laufzeit der \gls{bts} wechseln zu k�nnen, wird dieser in der Programmiersprache F\# entwickelt und als \gls{dll} gespeichert. Diese \gls{dll} kann einfach �ber eine \gls{gui} ausgew�hlt und f�r die aktuelle Berechnung herangezogen werden.\\
\\
Laut Aufwandssch�tzung entsteht ein Aufwand von insgesamt 480 h und Kosten in der H�he von \EUR{36.141}. Der geplante Projektzeitraum ist vom 14.11.2012 bis 10.04.2013.


\begin{center}
\begin{tabular}{ | l | p{2cm} | p{2cm} | p{2cm} | p{1.5cm} | p{2.8cm} | }
\hline 
\textbf{Version} & \textbf{Autor} & \textbf{QS} & \textbf{Datum} & \textbf{Status} & \textbf{Kommentar} \\  \hline
0.1 & Sochovsky & Pawlowsky & 22.09.2012 & draft & Erstversion \\ \hline
0.2 & Pawlowsky & Nagy & 22.09.2012 & draft & Produkt\-funktionen\\ \hline
0.3 & Sochovsky & Pawlowsky & 23.09.2012 & draft & Technische Machbarkeit \\ \hline
0.4 & Sochovsky & Nagy & 24.09.2012 & draft & Nutzwert\-analyse\\ \hline
0.5 & Sochovsky & Pawlowsky & 26.09.2012 & draft & Wirtschaftliche Machbarkeit \\ \hline
0.6 & Nagy & Pawlowsky & 27.09.2012 & draft & Finanz\-wirtschaft\-liche Grundlagen \\ \hline
0.7 & Pawlowsky & Sochovsky & 28.09.2012 & draft & FPA \\ \hline
0.8 & Sochovsky & Nagy & 30.09.2012 & draft & Sollzustand \\ \hline
0.9 & Sochovsky & Nagy & 06.10.2012 & draft & Projekt\-organisation \\ \hline
1.0 & Pawlowsky & Sochovsky & 07.10.2012 & draft & Aufwands\-absch�tzung \\ \hline
1.1 & Nagy & Sochovsky & 15.10.2012 & draft & Einleitung \\ \hline
1.2 & Nagy & Pawlowsky & 17.10.2012 & fertiggestellt & Management Summary, Nutzen \\ \hline
\end{tabular}
\end{center}