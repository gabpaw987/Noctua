\chapter{Ziele} \label{chapter:ziele}

\section{Musskriterien}

Es soll ein Algorithmus in einer popul�ren Sprache zur vollautomatischen Bestimmung von m�glichst profitablen Handelsaktionen (Trades) auf transparenten Handelsm�rkten, prim�r dem Aktienmarkt, entwickelt werden. Um dieses Ziel zu erreichen, sollen Indikatoren aus der technischen Analyse ausgeforscht und im Entwicklungsprozess bewertet werden, um in Folge dessen ein m�glichst profitables Handelssystem zu entwickeln. Zus�tzlich wird versucht, verschiedene Marktzust�nde, und die dadurch entstehenden Implikationen auf Kursbewegungen von b�rsennotierten Handelspapieren, von einander zu unterscheiden. Ein Marktzustand ist hier als unterscheidbares Verhalten der Kurse bzw. Kursbewegungen zu verstehen. Der Einsatz einer weit verbreiteten Programmiersprache ist hierbei besonders wichtig, da sich viele Menschen keine neue Sprache aneignen wollen, sondern es bevorzugen, die ihnen bekannten Sprachen weiter zu ben�tzen.\\
\\
Dabei sollen m�gliche Anwendungen von verschiedenen Moving-Averages (\glspl{ma}, gleitende Durchschnitte), Oszillatoren zur Support- und Resistance-Level Bestimmung und andere mehr oder weniger h�ufig genutzte Daten zur algorithmischen Entscheidungsfindung herangezogen werden.\\
\\
Eine Auswahl �blicher technischer Indikatoren und damit verbundene Handels\-strategien sollen auf Performance und Risiko �berpr�ft und m�gliche Optimierungen erkannt und umgesetzt werden.
Um diese Gr��en vergleichbar zu machen, soll Software entwickelt werden, die als Backtesting-Modul fungiert und anhand von historischen Kursverl�ufen relevante Kennzahlen und Messgr��en errechnet.\\
\\
M�rkte verhalten sich in unterschiedlichen Zeitperioden und unter anderen Randbedingungen unterschiedlich. Zeitweise so stark, dass Tradingstrategien, die zeitweise gut oder sogar sehr gut funktionieren, unter anderen Randbedingungen wesentlich niedrigere Ertr�ge einbringen oder sogar Verluste verursachen. Um genau das zu vermeiden und die Volatilit�t und damit das Risiko zu verringern, soll versucht werden, diese Randbedingungen zu bestimmen und somit spezielle Marktzust�nde zu identifizieren. \\

Um die wichtigsten Ziele des Projektes zusammenzufassen, k�nnen folgende Punkte aufgef�hrt werden:

\begin{itemize}
	\item{Erkennen von unterschiedlichen Marktzust�nden}
	\item{Entwickeln eines \emph{parametrisierbaren} Algorithmus (Signalgenerators)\\
		Beispielweise soll der Algorithmus f�r unterschiedliche Marktzust�nde andere Parameters�tze anwenden oder die Strategie selbst adaptieren.}
	\item{Entwickeln von Software zur Pr�fung von Performance und Risiko (Backtesting)\\
		Um die Entwicklung dahingehend zu unterst�tzen, dass fundierte Entscheidungen und Optimierungen getroffen werden k�nnen, sowie die Ergebnisse der Forschung objektiv bewerten zu k�nnen, ist eine solche Software unerl�sslich.}
\end{itemize}

Die folgenden Leistungen m�ssen im Laufe des Projektes erf�llt werden und m�ssen sich, abh�ngig von den Ergebnissen, nicht zwangsweise in das Endprodukt einflie�en.\\

\begin{itemize}
	\item{Testen der Performance eines Handelssystems mit 2 \glspl{sma} und 2 \glspl{ema}, die durch Kreuzungen Entscheidungen treffen.}
	\item{Testen eines Handelssystems mit 3 beliebigen \glspl{ma}, die durch definierte Stellung zueinander Entscheidungen treffen.}
	\item{Testen der Auswirkungen der Integration von Support- und Resistance-Mechanismen in das Handelssystem.}
\end{itemize}

\section{Wunschkriterien}

Die Technologie soll prim�r f�r kurze Perioden (Intra-Day bzw. Short-Term) entwickelt werden, sollte sich jedoch ebenfalls w�hrend des Projektes eine Eignung f�r l�ngerfristige Strategien ergeben, w�re dies vorteilhaft.\\
\\
Neben Daten, welche wie die Kurse selbst, Reaktionen der M�rkte auf Ereignisse darstellen, sollen exemplarisch realwirtschaftliche Daten auf ihre Eignung zur Verbesserung eines solchen Handelssystems herangezogen werden. Falls sich eine eindeutige Eignung feststellen l�sst, soll dies als Entscheidungsfaktor im System eine Rolle spielen.\\

\section{Abgrenzungskriterien}

Teil des fertigen Produktes ist ein entwickeltes Handelssystem, das algorithmisch umgesetzt wurde. Dazu werden vorhandene Indikatoren und Strategien aufgegriffen und mithilfe der \gls{bts} quantifiziert. Es ist hingegen \emph{nicht} vorgesehen, komplett neue Indikatoren selbst zu entwickeln.\\
\\
Im Projektverlauf wird der Algorithmus anhand von realen historischen Daten getestet und dessen Performance �berpr�ft. Die Integration oder Entwicklung einer Trading-Umgebung, die die aktuellen Kaufentscheidungen des Algorithmus umsetzt, ist \emph{nicht} Bestandteil des Projektes.