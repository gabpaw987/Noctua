\chapter*{Kurzfassung}

H�ufiges Handeln mit Aktien und anderen Wertpapieren, oft mit mehreren Transaktionen
t�glich, setzt sowohl Echtzeit-Kursdaten, als auch und Computersysteme, die diese Daten aufbereiten und auswerten, voraus.
Es existieren bereits eine Unmenge von Methoden zur Analyse von Kursbewegungen und zur Bestimmung von g�nstigen Handelszeitpunkten,
die als Softwarealgorithmus implementiert werden k�nnen.\\

Diese Arbeit befasst sich mit der Entwicklung und Programmierung einer algorithmischen
Handelsstrategie f�r Wertpapiere auf der Basis von Verfahren und Indikatoren der Technischen Analyse
sowie der Charttechnik.
Dazu werden diverse Indikatoren, wie der \gls{rsi}, \gls{adx} oder die Bollinger B�nder und Methoden, wie die Bildung von Handelssystemen mit
\glspl{ma} oder Regressionen beschrieben, erprobt und evaluiert. Das Resultat sind zwei unterschiedliche Ans�tze, die schrittweise
erweitert wurden und die, passende Parameter vorausgesetzt, in der Lage sind profitabel zu handeln --- ohne menschliches Eingreifen.\\

Um die Evaluation solcher Algorithmen allgemein und des entwickelten Algorithmus im speziellen
zu vereinfachen, wurde eine Software zur �berpr�fung der Performance entwickelt, die Entscheidungen
aufgrund von historischen Daten evaluiert und im Folgenden als \gls{bts} bezeichnet wird.
Durch eine grafische Benutzeroberfl�che, intuitive Bedienung und informative Charts wird diese au�erdem allen Anspr�chen
an Benutzerfreundlichkeit und Usability gerecht.