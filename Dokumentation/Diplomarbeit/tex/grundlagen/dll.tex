% !TEX root = ../../Noctua_Diplomarbeit.tex

\subsection{Dynamic Link Library} \label{dll}

Alle .Net-Sprachen bieten die Funktionalit�t Bibliotheksprojekte zu entwickeln. Normalerweise werden ausf�hrbare Programme entwickelt, w�hrend bei einem DLL-Projekt oder einer Bibliotheksanwendung nur eine Funktionalit�t implementiert werden muss. \cite{dllmsdn1}\\ \\
Eine DLL enth�lt Code und Daten. Man kann diese von mehreren Programmen gleichzeitig verwenden. Ihre Funktionalit�t ist allerdings nicht begrenzt, das bedeutet man kann mit ihnen genau die gleichen Ergebnisse erzielen, wie mit einer ausf�hrbaren Software, allerdings agiert eine DLL nicht von alleine. \\ \\
Prinzipiell werden DLLs programmiert um eine Software in unterschiedliche Komponenten (oder auch Modulen) zu trennen. Der Vorteil bei der Verwendung von DDLs ist die Einbindung in ein bestehendes Softwarekonstrukt zur Laufzeit, dass erh�ht die Erweiterbarkeit und Skalierbarkeit von Softwareprodukten. \cite{dllmsdn2} \\ \\
Bei einem Software-Update-Vorgang k�nnen beispielsweise nur die ver�nderten Komponenten vertauscht werden. Ein erneutes Installieren der Software ist somit nicht n�tig und kann vermieden werden. Im abstrakteren Geschehen bedeutet das, dass der Programmkern eine Ver�nderung der Funktionen nicht bemerkt und auch kein Teil der Erweiterung oder der Fehlerbehebung sein muss. Achtet man darauf komplexere Berechnungen in auswechselbaren Komponenten zu benutzen und die Komplexit�t des Kernprodukts gering zu halten, kann man ein Softwareprodukt leichter warten und weiterentwickeln. \cite{dllmsdn1}
\subsubsection{DLL-Typen}
Es gibt zwei M�glichkeiten die Funktionen einer DLL aufzurufen:
\begin{itemize}
	\item Load-Time Dynamic Linking
	\item Run-Time Dynamic Linking
\end{itemize}
Beim Load-Time Dynamic Linking wird bereits zur Kompilierzeit ein explizites Interface bereitgestellt, um die bereitgestellten Methoden explizit aufrufen zu k�nnen.\\ \\
Das Run-Time Dynamic Linking verwendet dynamische Methodenaufrufe, das bedeutet das auch beim Kompiliervorgang noch nicht klar ist welche Methoden die DLL bereitstellt. Die DLL wird erst w�hrend der Laufzeit in die Software eingebunden. \cite{dllmsdn2}