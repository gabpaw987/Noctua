% !TEX root = ../../Noctua_Diplomarbeit.tex

\subsection{Dynamic Link Library} \label{dll}

Alle .NET-Sprachen bieten die Funktionalit�t, Bibliotheksprojekte zu entwickeln. Normalerweise werden ausf�hrbare Programme entwickelt, w�hrend bei einem \gls{dll}-Projekt oder einer Bibliotheksanwendung nur eine Funktionalit�t implementiert werden muss. \cite{dllmsdn1}\\ \\
Eine DLL enth�lt Code und Daten. Man kann diese von mehreren Programmen aus gleichzeitig verwenden. Ihre Funktionalit�t ist allerdings nicht auf Ein- und Ausgaben begrenzt, mit Programmbibliotheken lassen sich die gleichen Ergebnisse erzielen wie mit einer ausf�hrbaren Software, allerdings agiert eine \gls{dll} nicht von alleine und ist nicht ausf�hrbar. \\ \\
Prinzipiell werden \gls{dll}s programmiert um eine Software in unterschiedliche Komponenten (oder auch Modulen) zu trennen. Der Vorteil bei der Verwendung von \gls{dll}s ist die Einbindung in ein bestehendes Softwarekonstrukt zur Laufzeit, dass erh�ht die Erweiterbarkeit und Skalierbarkeit von Softwareprodukten. \cite{dllmsdn2} \\ \\
Bei einem Software-Update-Vorgang k�nnen beispielsweise nur die ver�nderten Komponenten vertauscht werden. Ein erneutes Installieren der Software ist somit nicht n�tig und kann vermieden werden. Im abstrakteren Geschehen bedeutet das, dass der Programmkern eine Ver�nderung der Funktionen nicht bemerkt und auch kein Teil der Erweiterung oder der Fehlerbehebung sein muss. Achtet man darauf, komplexere Berechnungen in auswechselbaren Komponenten zu benutzen und die Komplexit�t des Kernprodukts gering zu halten, kann man ein Softwareprodukt leichter warten und weiterentwickeln. \cite{dllmsdn1}
\subsubsection{DLL-Typen}
Es gibt zwei M�glichkeiten, die Funktionen einer DLL aufzurufen:
\begin{itemize}
	\item Load-Time Dynamic Linking
	\item Run-Time Dynamic Linking
\end{itemize}
Beim Load-Time Dynamic Linking wird bereits zur Kompilierzeit ein explizites Interface bereitgestellt, um die bereitgestellten Methoden explizit aufrufen zu k�nnen.\\ \\
Das Run-Time Dynamic Linking verwendet dynamische Methodenaufrufe, das bedeutet, dass auch beim Kompiliervorgang noch nicht klar ist welche Methoden die DLL bereitstellt. Die \gls{dll} wird erst w�hrend der Laufzeit in die Software eingebunden. \cite{dllmsdn2}