\section{LINQ}

\gls{linq} stellt eine Spracherweiterung zu .NET dar, die mit dem .NET-Framework 3.5 und der Visual Studio Version 2008 hinzugef�hgt wurde. \gls{linq} erm�glicht es �ber ein neues Abstraktionsmodell aus vielen verschiedenen Datenquellen �ber die selbe Syntax Daten abzufragen. Unterst�tzt werden bspw. \gls{xml}-Dokumente, Datenbanktabellen, Excel-Tabellen, herk�mmliche Objekte oder Auflistungen von Objekten aller Art, und vieles mehr. \cite{visualcsharp}\\
\\
Da der Zugriff auf unterschiedliche Datenquellen intern allerdings durchaus unterschiedlich ablaufen muss, wurden von Microsoft mehrere \gls{linq}-Implementierungen in das .NET-Framework integriert. Diese werden Provider genannt und existieren zum Beispiel f�r folgende Datenquellen \cite{visualcsharp}:

\begin{itemize}
\item \textbf{\gls{linq} to Objects} \\
	Mit dieser Implementierung lassen sich Auflistungen und Objekte manipulieren, die untereinander auch in Beziehung gesetzt werden k�nnen. Sie stellt damit das Fundament aller\gls{linq}-Abfragen dar.
\item \textbf{\gls{linq} to \gls{xml}} \\
	Diese Implementierung nutzt das .NET sprachinterne Abfrage-Framework f�r den Zugriff auf \gls{xml} im Arbeitsspeicher.
\item \textbf{\gls{linq} to \gls{sql}} \\
	Hiermit kann auf Microsofts hauseigenes Datenbanksystem \gls{sql} Server 2005 und 2008 zugegriffen werden.
\end{itemize} \cite{visualcsharp}

\subsection{Aufbau einer LINQ-Anweisung}

Optisch sieht der Aufbau einer \gls{linq}-Anweisung, dem einer \gls{sql}-\textit{SELECT}-Anweisung sehr �hnlich. Eine Abfrage zur Ermittlung aller Personen mit Alter �ber 30 Jahren und der R�ckgabe derer Alter und Namen in einer Ergebnisliste, w�rde als \gls{linq}-Anweisung in etwa so aussehen \cite{visualcsharp}:

\begin{verbatim}
var pers = from p in personen
		   where p.Alter > 30
		   select new { p.Name, p.Alter };
\end{verbatim} \cite{visualcsharp}

Genau die selbe Funktion, nur mit einer anderen Formulierung w�rde auch die folgende Abfrage erf�llen \cite{visualcsharp}:

\begin{verbatim}
var pers = personen
		  .Where( p => p.Alter > 30 )
		  .Select( p => new {p.Name, p.Alter });
\end{verbatim} \cite{visualcsharp}

Dabei ist es v�llig egal, ob \textit{personen} das Ergebnis einer Datenbankabfrage oder eine Liste von \textit{Person}-Objekten ist, mit \gls{linq} muss man nur eine der beiden gezeigten Abfrage-Varianten benutzen.  Weiters erm�glicht die Nutzung von \gls{linq} auch viele zus�tzliche Optionen, die das Suchergebnis weiter verfeinern. Es gibt bspw. die \gls{sql}-�hnlichen Funktionen, wie \textit{GroupBy} und \textit{Join}, es gibt allerdings auch noch viele mehr. \cite{visualcsharp}

\subsection{Interne Funktionsweise von LINQ-Abfragen}

\gls{linq}-Abfragen machen oft Gebrauch vom Prinzip der \textit{Delegates}. \textit{Delegate} ist eine relativ alte Technologie, die in der Programmiersprache C schon unter dem Namen Funktionszeiger bekannt war. Ein \textit{Delegate} ist ein Typ, der auf eine Methode zeigt. Das Wort ''Delegate'' kommt von ''Delegierter'' und wurde gew�hlt, da ein \textit{Delegate} wirklich einen Methodenaufruf an eine bestimmte andere Methode weiter leitet. Dies wird innerhalb von \gls{linq}-Abfragen ben�tigt, um die vom User �bergebenen Bedingungen in die Logik der intern genutzten Methoden integrieren zu k�nnen. Der wohl am meisten in Kombination mit \gls{linq}-Abfragen genutzte \textit{Delegate} ist der folgende \cite{visualcsharp}:

\begin{verbatim}
public delegate TResult Func<T, TResult>(T arg)
\end{verbatim} \cite{visualcsharp}

Dieser wurde bspw. im oberen Beispiel verwendet, um die Bedingungen der \textit{Where}- (\textit{p => p.Alter > 30}) und der \textit{Select}-Abfrage (\textit{p => new {p.Name, p.Alter }}) an die \gls{linq}-Logik zu �bergeben. Beide dieser Bedingungen werden hier als Methodenaufrufe bzw. Zeiger auf Methodenaufrufe mit einem R�ckgabewert an die innere Logik der \gls{linq} �bergeben. Das \textit{T} in \textit{Func<T, TResult>} beschreibt einen generischen Datentypen mit dem innerhalb der Abfrage gearbeitet werden soll. Hier k�nnen alle existierenden primitiven Datentypen �bergeben werden und die Logik der \gls{linq} kann problemlos damit arbeiten. \textit{TResult} beschreibt ebenfalls einen generischen Datentypen. Allerdings den der vom angezeigten Methodenaufruf zur�ckgegeben wird. Intern ist die Abfragelogik der \gls{linq} damit so organisiert \cite{visualcsharp}:

\begin{verbatim}
class Programm {
	static void Main(string[] args) {
		string[] arr = { "Peter", "Uwe", "Willi", "Udo" };
		GetShortNames(arr, name => name.Length < 4 );
		Console.ReadLine();
	}
	
	static void GetShortNames<T>(T[] names, Func<T, bool> getNames) {
		foreach (T name in names )
			if (getNames(name))
				Console.WriteLine(name);
	}
}
\end{verbatim} \cite{visualcsharp}

In diesem Beispiel wird der Methode \textit{GetShortNames} ein \textit{Delegate} �bergeben, der angibt welche Namen der Namensliste ausgegeben werden soll. Hierbei h�tte in der \textit{Main}-Methode auch jede beliebige andere Bedingung, wie zum Beispiel die Abfrage, ob der erste Buchstabe des Namens ein ''P'' ist (\textit{GetShortNames(arr, name => name[0] == 'P');}) und es w�re genauso der entsprechende Name ''Peter'' ausgegeben worden, ohne dass die Methode ge�ndert werden musste. \cite{visualcsharp}