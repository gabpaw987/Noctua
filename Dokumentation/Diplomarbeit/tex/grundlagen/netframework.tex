\subsection{Allgemein}

.NET ist ein Framework der Firma Microsoft, das eine Vielfalt an Sprachbibliotheken f�r eine gro�e Palette an Programmiersprachen zur Verf�gung stellt. Fr�her arbeiteten alle Programmiersprachen von Microsoft, wie bspw. C++, mit der WinAPI-32. Nun wurde dieses \gls{api} mit .NET durch ein sprachenunabh�ngiges, weitaus umfangreicheres Framework ersetzt. \cite{visualcsharp}\\
\\
Insgesamt wird die Nutzung des .NET-Frameworks von �ber 30 Programmiersprachen unterst�tzt. Die Richtlinien die eine Sprache einhalten muss, um als .NET-konform angesehen werden zu k�nnen, sind von Microsoft als \gls{cls} definiert worden. Zu den ber�hmtesten Vertretern solcher Sprachen Z�hlen C\#, F\#, Visual Basic .NET (VB.NET) und C++. \cite{visualcsharp}\\
\\
Zu den wichtigsten Features des .NET-Frameworks z�hlen die folgenden:
\begin{itemize}
\item \textbf{Objektorientierung} \\
	Das .NET-Framework ist zu 100\% objektbasiert. Das bedeutet, jegliche Elemente, sogar einfache Datentypen wie bspw. Integer, lassen sich auf Objekte zur�ckf�hren. Ja sogar die internen Zugriffe des Frameworks auf das darunterliegende Betriebssystem sind in Klassen gekapselt.
\item \textbf{Plattformunabh�ngigkeit} \\
	Anwendungen, die das .NET-Framework benutzen, werden �hnlich der \gls{jvm} erst zur Laufzeit in Maschinencode umgewandelt. Weiters ist die Spezifikation der von Microsoft benutzten Laufzeitumgebung, der sog. \gls{clr}, offen zug�nglich. Dadurch verlangt die Nutzung von .NET-Anwendungen also eine solche Umgebung, diese kann allerdings auch auf Plattformen portiert werden, die nicht Windows hei�en. So gibt es neben der propriet�ren Windows-Implementierung des .NET-Frameworks von Microsoft bspw. auch das \textit{Mono}-Projekt, mit dem .NET bereits erfolgreich auf Linux portiert werden konnte.
\item \textbf{Sprachenunabh�ngigkeit} \\
	Alle Komponenten des .NET-Frameworks k�nnen von jeder unterst�tzten Sprache problemlos verwendet werden. Das .NET-Framwork erm�glicht aber auch die Nutzung aller, in einer .NET-konformen Programmiersprache geschriebener Komponenten, in jeder anderen .NET-konformen Programmiersprache. So k�nnen zum Beispiel alle in C\# 2010 geschriebenen Klassen auch in F\# oder VB.NET genutzt und sogar abgeleitet werden.
\item \textbf{Speicherverwaltung} \\
	Die explizite Freigabe von nicht mehr ben�tigtem Speicher hat in der Vergangenheit schon zu vielen Problemen gef�hrt. Daf�r wurde mit dem .NET-Framework ein \textit{Garbage Collector} eingef�hrt, der dem Programmierer diese Aufgabe automatisch abnimmt. 
\item \textbf{Weitergabe} \\
	Auch die Weitergabe konnte deutlich vereinfacht werden. So k�nnen auf .NET basierende Sprachen einfach in eine .EXE- oder .\gls{dll}-Datei kompiliert und von jedem beliebigen Ordner aus ausgef�hrt werden. Eine .EXE-Datei ist dabei eine direkt ausf�hrbare Datei, deren Code bereits in einzelne Bytes umgewandelt wurde. \gls{dll}-Bibliotheken werden im folgenden Punkt erl�utert. \cite{visualcsharp}
\end{itemize}