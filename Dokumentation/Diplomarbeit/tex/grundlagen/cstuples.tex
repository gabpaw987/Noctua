\section{C\# Tuples}

Ein Tuple ist prinzipiell eine Ansammlung von Werten. Dabei k�nnen sie beliebig viele Werte mit verschiedenen Datentypen haben. Eben diese Anzahl und die dazugeh�rigen Datentypen m�ssen allerdings schon im Vorhinein festgelegt werden, wenn der Tuple erzeugt wir. Ein Tuple k�nnte bsow. so aussehen:

\begin{verbatim}
Tuple<int, bool> tuple = new Tuple<int, bool>(1, true);
\end{verbatim}

Dabei handelt es sich um einen sehr einfachen Tuple der lediglich einen Integer-Wert zu einem dazugeh�rigen Boolean-Wert speichert.\\
Der Tuple ist eine Klasse dessen Werte in ihm gespeichert werden. Deshalb muss er einen separaten Ort im Heap allokieren. Dies bedeutet, dass bspw. ein Tuple mit zwei Werte in der Erstellung wesentlich l�ngert ben�tigt als ein KeyValuePair, er aber sp�ter in der Nutzung eine deutlich h�here Performance aufweist. Au�erdem k�nnen durch diese Tatsache die Datentypen der Felder des Tuples nach dessen Erstellung absolut nicht mehr ge�ndert werden, was den Tuple eigentlich mehr einer struct �hneln l�sst.\\
\\
In der Praxis werden selten einzelne Tuples verwendet. Es ist allerdings oft von Vorteil eine Liste aus Tuples zu erzeugen, wenn man quasi eine Map (bzw. ein Dictionary) mit mehreren Werten oder anderen Datentypen ben�tigt. So k�nnen z.B. Aktienpreis-Bars als Tuple aus einem TimeStamp und Feldern f�r Open, High, Low und Close realisiert werden. Daraus kann dann eine Liste erstellt werden, um die Bars ad�quat speichern zu k�nnen.