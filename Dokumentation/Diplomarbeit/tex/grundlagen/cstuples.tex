\section{C\# Tupel}

Ein Tupel ist prinzipiell eine Ansammlung von Werten. Dabei k�nnen sie beliebig viele Werte mit verschiedenen Datentypen haben. Eben diese Anzahl und die dazugeh�rigen Datentypen m�ssen allerdings schon im Vorhinein festgelegt werden, wenn der Tupel erzeugt wird. Ein \textit{Tuple} in C\# k�nnte bspw. so aussehen \cite{cs-tuple}:

\begin{verbatim}
Tuple<int, bool> tuple = new Tuple<int, bool>(1, true);
\end{verbatim}

Dabei handelt es sich um einen sehr einfachen Tuple der lediglich einen Integer-Wert zu einem dazugeh�rigen Boolean-Wert speichert.\\
Der Tupel ist eine Klasse dessen Werte in ihm gespeichert werden. Deshalb muss er einen separaten Ort im Heap allokieren. \cite{cs-tuple} Dies bedeutet, dass bspw. ein Tupel mit zwei Werte in der Erstellung wesentlich l�nger ben�tigt als ein \textit{KeyValuePair}, er aber sp�ter in der Nutzung eine deutlich h�here Performance aufweist. \cite{tuple-performance} Au�erdem k�nnen durch diese Tatsache die Datentypen der Felder des Tupels nach dessen Erstellung absolut nicht mehr ge�ndert werden, was den \textit{Tuple} eigentlich mehr einer \textit{struct} �hneln l�sst. \cite{cs-tuple}\\
\\
In der Praxis werden selten einzelne Tupeln verwendet. Es ist allerdings oft von Vorteil eine Liste aus Tupeln zu erzeugen, wenn man quasi eine \textit{Map} (bzw. ein \textit{Dictionary}) mit mehreren Werten oder anderen Datentypen ben�tigt. So k�nnen z.B. Aktienpreis-Bars als Tupel aus einem \textit{TimeStamp} und Feldern f�r Open, High, Low und Close realisiert werden. Daraus kann dann eine Liste erstellt werden, um die Bars ad�quat speichern zu k�nnen.