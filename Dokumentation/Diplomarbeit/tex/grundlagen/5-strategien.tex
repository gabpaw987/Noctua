% !TEX root = ../../Noctua_Diplomarbeit.tex

\subsection{Handelsstrategien}



\subsubsection{Strategien mit MAs}

Eine simples Tradingmodell basiert auf einer einfachen �berkreuzung eines \gls{ma} �ber den Preis. Es wird dabei angenommen, dass ein Aufw�rtstrend eingesetzt hat, wenn der Preis �ber den \gls{ma} steigt, da der Kurs begonnen hat, schneller als der Durchschnitt zu steigen. Umgekehrt wird angenommen, dass bei einem Abfall des Kurses unter den \gls{ma} ein Abw�rtstrend folgt und daher wird ein Short-Signal generiert. Eine zus�tzliche Sicherheit ist gegeben, wenn der \gls{ma} selbst in die erwartete Kursrichtung dreht.

Dieser simple Algorithmus hat einige Nachteile. Es ist zu jedem Zeitpunkt ein Signal gegeben, was bedeutet, dass immer entweder eine Long- oder eine Short-Position gehalten wird. Au�erdem verliert diese Strategie bei langen \glspl{ma} nach der Trendumkehr wieder viel vom Gewinn, da das Gegensignal erst sp�t generiert wird. Kurze \glspl{ma} erzeugen im Gegensatz oft Fehlsignale.\\

Eine h�ufiger verwendete Variante zur Signalgenerierung mit \glspl{ma} wird \emph{Double Crossover Method} genannt. Dabei kommen zwei unterschiedlich lange \glspl{ma} zum Einsatz, wobei ein Signal erzeugt wird wenn sich beide schneiden. Kreuzt der kurze \gls{ma} den l�ngeren entsteht ein Kaufsignal, auch \emph{Golden Cross} genannt, und vice versa. Diese Variante erzeugt weniger Fehlsignale als die direkte Verwendung des Preises, hinkt dem Markt daf�r aber auch st�rker hinterher. Die L�nge der Durchschnitte h�ngt wie immer sowohl vom Handelszeitraum und der gew�nschten Signalanzahl als auch vom Markt ab.

Dieses System kann noch um einen weiteren \gls{ma} erweitert werden. Der Einsatz dreier Durchschnitte, oder \emph{Triple Crossover Method}, verfeinert die Signalgenerierung nochmals. Einen beginnenden Aufw�rtstrend ist dann vorhanden, wenn der kurze \gls{ma} �ber dem mittleren liegt. Ein vollst�ndiges Kaufsignal entsteht sobald der kurze �ber dem mittleren, und jener wiederum �ber dem langen \gls{ma} notiert. Eine umgekehrte Anordnung ist als Verkaufssignal anzusehen. Auf diese Art kann beispielsweise bei unklaren Signalen eine Neutralstellung (i.e. keine Aktien im Portfolio) eingenommen werden oder die Market-Exposure (i.e. Anzahl der Aktien) reduziert werden.
Normalerweise werden f�r solche Systeme \glspl{sma} verwendet, wobei aber besonders bei einem Double Crossover System die Anwendung eines \gls{ema},  \gls{dema} oder sogar \gls{tema} m�glich ist.