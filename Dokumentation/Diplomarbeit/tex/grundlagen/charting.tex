\section{Microsoft Chart Controls}

Die Microsoft Charts Controls Library ist eine sehr umfangreiche Bibliothek, die es erm�glicht in WinForms-Anwendungen oder auf Webseiten mit ASP.NET Charts und Diagramme zu zeichnen. Die M�glichkeiten sind derma�en umfangreich, dass es im Folgenden nur Sinn macht, die Funktionen zu erl�utern, die f�r die Darstellung finanzieller Charts erheblich sind. Zu den Hauptfeatures z�hlen daher \cite{msdn-charting}:

\begin{itemize}
\item \textbf{Charttypen} \\
	Es werden 35 verschiedene Charttypen unterst�tzt, dabei alles von normalen Liniencharts bis zu Bar- und Candlestick-Charts zur Darstellung von Aktiendaten.
\item \textbf{Skalierbarkeit} \\
	Es k�nnen unendlich viele Daten, Chart-Areas (Bereiche innerhalb eines Charts), Bemerkungen (z.B. Pfeile), etc. innerhalb eines Charts angezeigt werden. Daher k�nnen bspw. Aktiendaten und deren \glspl{ma} in einer ChartArea angezeigt werden. Zus�tzlich k�nnte man eine weitere ChartArea erg�nzen um bspw. einen \gls{macd}-Verlauf direkt unter den Aktiendaten zu zeichnen.
\item \textbf{Daten} \\
	Es werden Unmengen an M�glichkeiten zur Verwaltung der Daten innerhalb eines Charts, wie z.B. Data Binding, zur Verf�gung gestellt. Au�erdem k�nnen die Daten des Charts exportiert, oder zum Speichern serialisiert werden. 
\item \textbf{Aussehen} \\
	Weiters k�nnen sowohl zweidimensionale als auch dreidimensionale Charts erstellt werden, die alle sehr simpel und sch�n dargestellt werden. Die k�nnen zudem auch zur Laufzeit noch weiter manipuliert werden und die WinForms-Version bietet sogar M�glichkeiten zum interaktiven Zoomen und Scrollen f�r den User. (Auf Webseiten mit ASP.NET werden keine interaktiven Funktionen f�r den Benutzer der Website angeboten.) 
\end{itemize} \cite{msdn-charting}

\subsection{Nutzung mit WPF}

Leider ist momentan noch keine Version f�r die Verwendung mit \gls{wpf} erh�ltlich, man kann diese Einschr�nkung allerdings umgehen indem man einen sog. \textit{WinFormsHost} erstellt. Dies ist ein \gls{wpf}-\gls{gui}-Element, das WinForms-Elemente so kapselt, dass sie in \gls{wpf} genutzt werden k�nnen. Dieses Element mit einem \textit{MSChart} kann so erstellt werden:

\begin{verbatim}
<WindowsFormsHost Name="WfHost" Grid.Row="0">
	<MSChart:Chart x:Name="MyWinformChart">
	    <MSChart:Chart.ChartAreas>
	        <MSChart:ChartArea Name="MainArea"/>
	    </MSChart:Chart.ChartAreas>
	</MSChart:Chart>
</WindowsFormsHost>
\end{verbatim}

Hier wurde au�er dem primitiven \textit{MSChart} auch gleich eine \textit{ChartArea} hinzugef�gt. Eine \textit{ChartArea} ist ein Bereich innerhalb des Charts (hier der gesamte Bereich des Charts), in dem eine Funktion gezeichnet werden kann. Damit dieses WinFormsHost-Element allerdings funktioniert m�ssen f�r das gesamte \textit{Window} noch folgende Namespaces erg�nzt werden (diese Assemblies m�ssen nat�rlich auch als Verweis zum Projekt hinzugef�gt werden):

\begin{verbatim}
xmlns:wf="clr-namespace:System.Windows.Forms;assembly=System.Windows.Forms"
xmlns:MSChart="clr-namespace:System.Windows.Forms.DataVisualization.Charting;assembly=System.Windows.Forms.DataVisualization"
xmlns:wfi="clr-namespace:System.Windows.Forms.Integration;assembly=WindowsFormsIntegration"
\end{verbatim}

\subsection{Darstellung einfacher Graphen}

Nachdem eine leere \textit{ChartArea}, in die ein Graph gezeichnet werden kann, bereits im \gls{xaml} erzeugt wurde, muss man im C\#-Code nur noch das \textit{Chart} suchen, in das gezeichnet werden soll und eine \textit{Series} erstellen. In diese Series werden dann die Daten eingespeist, die gezeichnet werden soll. Wenn man nun auch noch einen Typ f�r den Graphen konfiguriert wird dieser auch schon in der \textit{ChartArea} angezeigt. Um bspw. eine Sinusfunktion darzustellen m�sste man nun folgenden Code schreiben: \cite{msdn-charting}

\begin{verbatim}
//Lookup des bereits im XAMl-Code definierten Charts
Chart chart = this.FindName("MyWinformChart") as Chart;

//Erstellen der Series
Series sinus = new Series("Sinus");

//Berechnung einer Sinuskurve und speichern der
//errechneten Daten in die Series
for (double i = 0; i <= 7.5; i += 0.2)
	sinus.Points.AddXY(i, Math.Sin(i));

//Definieren des Zeichentyps als normale Linie
sinus.ChartType = SeriesChartType.FastLine;
 
// Hinzufuegen des Graphs zum Chart
chart.Series.Add(sinus);
\end{verbatim} \cite{msdn-charting}

In diesem Beispiel wurde jeder Punkt des Sinus einzeln berechnet und zur \textit{Series} hinzugef�gt. Dies k�nnte man nat�rlich auch mit jeder anderen Datenquelle wie zum Beispiel einer Liste machen, in dem man jeden Wert einzeln hinzuf�gt. \cite{msdn-charting}

\subsection{Finanzielle Berechnungen}

Die Microsoft Chart Controls Library bietet allerdings nicht nur M�glichkeiten zur einfachen Darstellung von Daten. Denn sie beinhaltet ein Framework aus �ber 50 veschiedenen statistischen und finanziellen Formeln zur automatischen Berechnung und Darstellung spezieller Werte. \cite{msdn-charting}\\
\\
Die wichtigsten Formeln in der \textit{FinancialFormula}-Sammlung der MS Chart Controls sind:

\begin{itemize}
\item Bollinger B�nder
\item \gls{cci}
\item diverse \glspl{ma}
\item \gls{macd}
\end{itemize} \cite{msdn-charting}

Im Code w�rde der Einsatz von \textit{FinancialFormula} so aussehen:

\begin{verbatim}
chart.DataManipulator.FinancialFormula(FinancialFormula.WeightedMovingAverage, 90, "Data", "FinFor");
\end{verbatim} \cite{msdn-charting}

Dieser Aufruf erzeugt einen \gls{ma} der L�nge 90. Dazu holt er sich die Daten aus der Series ''\textit{Data}'', berechnet den \gls{ma} und schreibt die errechneten Daten in die \textit{Series FinFor}. Damit diese Berechnung m�glich ist muss die \textit{Series FinFor} bereits zuvor erzeugt und die \textit{Series Data} ebenfalls bereits zuvor erzeugt auch wirklich mit Daten gef�llt sein. \cite{msdn-charting}