\subsection{Allgemein}

Die Programmiersprache C\# wurde von der Firma Microsoft entwickelt und gilt als einer der wichtigsten Sprachen, die das .NET-Framework benutzen. C\# ist eine objektorientierte Programmiersprache mit einer fundamentalen Sprachsyntax. Mit diesen Eigenschaften eignet sie sich auch perfekt f�r die Nutzung von .NET. \cite{visualcsharp}\\
\\
Die Sprachsyntax von C\# ist der von Java sehr �hnlich. C\# wurde erstmals mit dem Ziel entwickelt, eine bessere, funktionsreichere Sprache zu entwickeln, die Java abl�sen k�nnte, trotzdem allerdings dessen Vorteile nutzt. Auch die allgemeine Struktur einer Klasse sowie der Aufbau simpler Anweisungen (if, for, while, etc.) sind quasi ident zu ihren �quivalenten in Java. \cite{visualcsharp}\\
\\
Eine C\#-Anwendung kann grunds�tzlich als Konsolenanwendung oder mit einer graphischen Oberfl�che (\gls{gui}) ausgef�hrt werden. Zur Realisierung einer solchen \gls{gui} werden ebenfalls mehrere Mechanismen zur Verf�hung gestellt. Zum einen gibt es die mittlerweile veraltete direkte M�glichkeit der Realisierung mit WinForms, einer API, die direkt in die Sprache integriert ist. Zum anderen hat Microsoft die \gls{wpf} entwickelt, die im folgenden Abschnitt \ref{wpf} genauer erl�utert wird. \cite{visualcsharp} \\
\\
Im folgenden sollen nun alle wichtigen Technologien der Sprache C\# beschrieben werden, die im Zuge dieses Projektes zum Einsatz gekommen sind.