W�hrend trendarmen Phasen kann man mit der beschriebenen Methode (siehe \ref{regr}) keine Signale generieren. Die Steigungen der l�ngerfristigen Regressionsberechnungen liegen zwischen $0.5$ und $-0.5$. Aufgrund der kurzfristigen Berechnungen kann man oft, beziehungsweise zu oft, einen Preiseinsturz "`erkennen"'. Offensichtlich kann man den generierten Signalen in solchen Marktzust�nden nicht trauen, die Regressionsberechnung funktioniert w�hrend trendstarken Phasen bedeutend besser. Eine M�glichkeit diesen Signalgeber zu verbessern und zu erweitern, w�re die Verwendung der Fading-Strategie (siehe \ref{subsection:fading}). Durch die Erweiterung durch diese Methode, kann der Signalgeber mit sinnvollen Handlungen w�hrend der bereits genannten schw�cheren Trendphasen unterst�tzt werden. Eine M�glichkeit solche Trendphasen zu erkennen bieten Regressionsberechnungen mit L�ngen zwischen  $20$ und  $40$. Sind die Steigungen dieser, unter $0.5$ und �ber $-0.5$ kann man den momentanen Marktzustand als trendarm kennzeichnen. Diese Schwellenwerte lassen sich allerdings beliebig anpassen und sind nicht bindend. Ebenso ist eine Trendst�rke-Berechnung mit Hilfe des ADX (siehe \ref{subsection:adx})
