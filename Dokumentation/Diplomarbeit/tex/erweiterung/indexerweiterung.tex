
Im Zusammenhang des Noctua-Algorithmus mit Indikator-Kombination wurde im Bereich der
Marktzustandsadaption ein Ansatz erw�hnt, der Kurse mit relevanten Indizes abgleicht,
um daraus Schl�sse f�r die Entwicklung des Kurses selbst zu ziehen. (siehe \ref{subsubsection:marktzustandsadaption})
Obwohl dieser Ansatz im Rahmen dieses Projektes nicht weiter verfolgt wurde, w�re eine zuk�nftige Erweiterung
des Ansatzes in mehrere Richtungen m�glich.\\

Einerseits k�nnte die Bestimmung des Zusammenhangs mit unterschiedlichen Mitteln bewerkstelligt werden,
z.B. �ber �hnliche Steigungen von Regressionsgeraden zu unterschiedlichen Zeitpunkten oder �ber die
Berechnung der Abweichungen von \gls{rsi}-Werten.\\

Andererseits k�nnten ebenso andere Zusammenh�nge erforscht werden, die auf gleiche Weise funktionieren.
Ein Aktienindex k�nnte beispielsweise durch den Kurs eines Rohstoffes ersetzt werden, von dem eine
Firma hochgradig abh�ngig sein k�nnte.

Schlussendlich w�re es au�erdem m�glich, dass sich der Zusammenhang in vielen F�llen invers der Annahme verh�lt,
indem nicht eine Aktie dem gesamten Markt folgt, sondern der Markt einer f�hrenden Aktie, die Trends f�r den gesamten
Markt ausl�st. 