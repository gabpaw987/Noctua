\lstset{style=sharpc}
\subsection{Persistente Sicherung der Einstellungen} \label{Einstellungen-sichern}

Die \gls{bts} bietet auf dem Settings-Tab eine Vielzahl an Einstellungen, die ein Benutzer normalerweise einmal w�hlt und danach nur noch selten ver�ndert. Deshalb werden die gew�hlten Einstellungen bei jedem Schlie�en der \gls{bts} persistent in einer Anwendungskonfigurationsdatei (Siehe \ref{akd} ''Anwendungskonfigurationsdatei'') gespeichert, um beim n�chsten Start der Software wieder geladen werden zu k�nnen.\\
\\
Folgende Daten werden auf diese Art und Weise persistent gespeichert:

\begin{easylist}[itemize]
& \textbf{General-Settings}
	&& Pfad zum Algorithmus- und Daten-File als \textit{string}s
	&& Gew�hlte Berechnungsperiode als \textit{DateTime}s
& \textbf{Orders-Settings}
	&& Startkapital als \textit{string}
	&& Absolute und relative Transaktionsgeb�hr sowie der Price Premium Prozentsatz als \textit{string}s
	&& Slider-Werte zur Beschreibung der Order Gr��en sowie die Round Lot Gr��e als \textit{int}
& \textbf{Chart-Settings}
	&& Serialisierte \textit{StackPanel}s zum Wiederaufbau der konfigurierten Indikatoren als \textit{StringCollection}
\end{easylist}

Generell werden all diese Einstellungen direkt aus dem ViewModel in die Anwendungskonfigurationsdatei abgelegt. Die einzige Ausnahme davon stellen die Chart-Settings dar, da diese im ViewModel als \textit{List<StackPanel>} gespeichert werden. Anwendungskonfigurationsdateien unterst�tzten allerdings nur ein gewisses Spektrum an Standarddatentypen, zu denen \textit{List<StackPanel>} leider nicht geh�rt. Daher wurde hier auf eine \textit{StringCollection} ausgewichen.\\
\\
Dazu m�ssen klarerweise die einzelnen \textit{StackPanel}s in der Liste in einzelne \textit{string}s umgewandelt werden, die in der \textit{StringCollection} gespeichert werden k�nnen. Dies funktioniert durch Serialisierung jedes einzelnen Panels. Leider die benutzten \gls{wpf}-\gls{gui}-Elemente allerdings nicht als serialisierbar gekennzeichnet und k�nnen nur �ber einen speziellen \textit{XamlWriter} in ein \gls{xaml}-Format serialisiert werden, dass von einem entsprechenden \textit{XamlReader} wieder ausgelesen und zur�ck in echte \textit{StackPanel}s umgewandelt werden kann. Weiters wird in jedem der \textit{StackPanel}s auch ein \textit{ColorPicker} aus dem Extended WPF Toolkit \cite{extended-wpf-toolkit} von Xceed benutzt. Dieser besitzt leider nicht die M�glichkeit �ber einen \textit{XamlWriter} serialisiert zu werden. Dieses Problem wurde umgangen, indem der \textit{ColorPicker} vor der Serialisierung aus dem \textit{StackPanel} entfernt wird. Zur \textit{StringCollection} wird dann ein weiterer \textit{string} hinzugef�gt, der die ausgew�hlte Farbe des \textit{ColorPicker}s speichert, damit diese beim Wiederaufbau der \textit{StackPanel}s einfach neu Initialisiert und auf die entsprechend zuvor gew�hlte Farbe gesetzt werden kann. Anschlie�end wird der \textit{ColorPicker} nat�rlich auch wieder zum \textit{StackPanel} hinzugef�gt, damit er nicht in der Anzeige der Chart-Settings fehlt. Die hierzu verwendete Methode sieht so aus:

\begin{lstlisting}[label=XML-Serialisierung der IndicatorStackPanels,caption=XML-Serialisierung der IndicatorStackPanels]
public StringCollection storeIndicatorStackPanels(List<StackPanel> stackPanels)
{
    StringCollection strings = new StringCollection();
    //Wenn StackPanels vorhanden sind, fortfahren
    if (stackPanels.Count != 0)
    {
    	//Das folgende soll f�r jedes StackPanel durchgefuehrt werden
        foreach (StackPanel sp in stackPanels)
        {
        	//Wenn es ein richtig initialisiertes StackPanel ist, fortfahren
            if (sp.Children[sp.Children.Count - 1].GetType().IsAssignableFrom((new ColorPicker()).GetType()))
            {
            	//Speichern und Loeschen des nicht-serialisiertbaren ColorPickers
                ColorPicker cp = ((ColorPicker)sp.Children[sp.Children.Count - 1]);
                sp.Children.Remove(cp);
                //Serialisieren des restlichen StackPanels
                strings.Add(XamlWriter.Save(sp));
                //Speichern der Farbe des ColorPickers
                strings.Add(cp.SelectedColor.A + ";" + cp.SelectedColor.R + ";" + cp.SelectedColor.G + ";" + cp.SelectedColor.B);
                //Erneutes Hinzuf�gen des ColorPickers zum StackPanel
                sp.Children.Add(cp);
            }
        }
    }
    return strings;
}
\end{lstlisting}

Die daraus resultierende \textit{StringCollection} kann nat�rlich auch in die Anwendungskonfigurationsdatei gespeichert werden. Beim Auslesen k�nnen die \textit{StackPanel}s wie folgt wieder aufgebaut werden:

\begin{lstlisting}[label=XML-Deserialisierung der IndicatorStackPanels,caption=XML-Deserialisierung der IndicatorStackPanels]
public List<StackPanel> restoreIndicatorStackPanels(StringCollection strings)
{
    List<StackPanel> newList = new List<StackPanel>();

	//Durchlaufen der StringCollection
    for (int i = 0; i < strings.Count; i += 2)
    {
    	//Deserialisieren des StackPanels
        newList.Add((StackPanel)XamlReader.Parse(strings[i]));

		//Neues Speichern der Listener zu jeder TextBox
        for (int j = 0; j < newList[i / 2].Children.Count; j++)
        {
            if (newList[i / 2].Children[j] is System.Windows.Controls.TextBox)
                ((System.Windows.Controls.TextBox)newList[i / 2].Children[j]).PreviewTextInput += NumericOnly;
        }

		//Erzeugen eines neuen ColorPickers und Setzen der gespeicherten Farbe
        string[] argb = strings[i + 1].Split(';');
        ColorPicker cp = this.AddColorPicker();
        cp.SelectedColor = System.Windows.Media.Color.FromArgb(Convert.ToByte(argb[0]), Convert.ToByte(argb[1]), Convert.ToByte(argb[2]), Convert.ToByte(argb[3]));
        newList[i / 2].Children.Add(cp);

		//Hinzuf�gen des Listeners f�r den Remove-Button
        ((System.Windows.Controls.Button)newList[i / 2].Children[0]).Click += RemoveIndicatorButton_Click;
    }

    return newList;
}
\end{lstlisting}