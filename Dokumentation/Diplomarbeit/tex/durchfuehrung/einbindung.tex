\subsection{Einbindung des Algorithmus}

Da die \gls{bts} nat�rlich nicht nur einen Algorithmus testen k�nnen soll, ist es wichtig das die Algorithmen nicht direkt im Code verankert sind, sondern zur Laufzeit geladen werden k�nnen. Dazu kann der Benutzer einen Pfad angeben, von dem der Algorithmus in Form einer \gls{dll}-Datei (Siehe \ref{dll} ''Dynamic Link Library'') gelesen werden kann. Damit die \gls{bts} den Algorithmus aber auch ausf�hren kann, muss dieser zuerst in die lokale Dom�ne des Benutzers bzw. der \gls{bts} geladen werden. \\
Dazu gibt es zwei M�glichkeiten:

\begin{itemize}
\item \textbf{\textit{LoadFrom}} \\
	Die \textit{LoadFrom}-Methode erm�glicht es eine Assembly (bspw. unser \gls{dll}-File) in die lokale Dom�ne zu laden und Methoden dieser Assembly aufzurufen. Ein kleines Problem hierbei ist allerdings, dass man die geladene Assembly in der Dom�ne nur sehr aufwendig �berschreiben oder l�schen kann, daher ist sie f�r unsere Anwendung nicht perfekt geeignet.
\item \textbf{\textit{LoadFile}} \\
	Die \textit{LoadFrom}-Methode erf�llt nahezu den gleichen Zweck, wie die \textit{LoadFile}-Methode, nur dass man hier einfach eine zweite Assembly in die Dom�ne laden kann und die erste wird �berschrieben. Daher wird in der \gls{bts} diese Methode genutzt.
\end{itemize}

Der komplette Codeausschnitt zum Laden eines Algorithmus sieht nun wie folgt aus:

\begin{verbatim}
//Laden des Assemblys
Assembly assembly = Assembly.LoadFile(this.mainViewModel.AlgorithmFileName);
//Laden des Assemblys in die Dom�ne
AppDomain.CurrentDomain.Load(assembly.GetName());
//Speichern der Klasse, zum spaeteren Ausfuehren von Methoden
Type t = assembly.GetType("Algorithm.DecisionCalculator");

//Uebergabeparameter der Methode, die aufgerufen wird,
//muessen als Objektarray uebergeben werden
Object[] oa = { this.mainViewModel.BarList, this.mainViewModel.Signals };
//Invokieren der Methoden mit den Uebergabeparametern
t.GetMethod("startCalculation").Invoke(null, oa);
\end{verbatim}

Durch Aufruf der \textit{startCalculation}-Methode des Algorithmus werden die Signale in die �bergebene Liste aus dem ViewModel gespeichert. Diese k�nnen von dort anschlie�end f�r die Berechnungen benutzt werden. Hierzu ist noch wichtig zu erw�hnen, dass von der \gls{bts} an den letzten Bar noch ein Signal mit dem Wert 0 gesetzt wird, damit sp�ter in den Berechnungen alle noch offenen Positionen aufgel�st werden und man die fiktive letztendliche Performance betrachten kann, die in diesem Fall aufgetreten w�re.

\subsection{Auslesen der Aktien-Preisdaten}

Die Aktein-Preisdaten werden in Form von Bars �ber einen bestimmten Zeitraum erwartet. Ob es sich hierbei um Minute-, Daily- oder andere Bars handelt ist hierbei grunds�tzlich egal. Sie m�ssen lediglich in Form einer \gls{csv}-Datei mit folgendem Format gespeichert werden:

\begin{verbatim}
Bar,Date,Time,Open,High,Low,Close
1,01/02/90,00:00,8.8125,9.375, 8.75,9.3125
2,01/03/90,00:00,9.375, 9.50,9.375,9.375
3,01/04/90,00:00,9.375,9.6875,9.3125,9.40625
...
\end{verbatim}

Das Datum muss im Format MM/DD/YY und die Uhrzeit im Format hh:mm gespeichert werden. Die Dezimalstellen von Open, High, Low und Close m�ssen durch einen Punkt und ohne Tausendertrennzeichen angegeben werden. Es k�nnen nat�rlich unendlich viele Bars in der Datei gespeichert werden, die Berechnungen dauern dadurch einfach entsprechend l�nger. Weiters wird die erste Zeile nicht ber�cksichtigt, da hier meist Header-Informationen gespeichert sind. Befindet sich hier auch schon ein Bar, wird dieser ignoriert. Noch genauere Informationen hierzu k�nnen dem Benutzerhandbuch von Noctua entnommen werden\\
\\
Ist der Pfad in der \gls{bts} richtig gew�hlt und die Berechnung wird gestartet, wird das \gls{csv}-File in der \gls{bts} ausgelesen. Dies geschieht mit Hilfe der \gls{linq}-Technologie (Siehe \ref{linq} ''LINQ'') und sieht im Code wie folgt aus:

\begin{verbatim}
public static IEnumerable<Tuple<DateTime, decimal, decimal, decimal, decimal>> EnumerateExcelFile(string filePath, DateTime startDate, DateTime endDate)
{
    // Geht alle Zeilen durch und ueberspringt den Header
    return from line in File.ReadLines(filePath).Skip(1)
           select line.Split(',')
               into fields
               //Parst die das Datum und die Uhrezit und speichert es in den Tupel
               let timeStamp = DateTime.ParseExact(fields[1] + " " + fields[2], "MM/dd/yy HH:mm", new CultureInfo("en-US"))
               //Speichert alle anderen Werte in den Tupel und beruecksichtigt Dezimalzahlen 
               let open = decimal.Parse(fields[3], CultureInfo.InvariantCulture)
               let high = decimal.Parse(fields[4], CultureInfo.InvariantCulture)
               let low = decimal.Parse(fields[5], CultureInfo.InvariantCulture)
               let close = decimal.Parse(fields[6], CultureInfo.InvariantCulture)
               //Liest nur die Bars im gew�hlten Bereich aus
               where timeStamp.Date >= startDate.Date && timeStamp.Date <= endDate.Date
               //Erzeugt den Tupel und speichert ihn in das Rueckgabe-IEnumerable
               select new Tuple<DateTime, decimal, decimal, decimal, decimal>(timeStamp, open, high, low, close);
}
\end{verbatim}