\subsection{Berechnung der Performancedaten}

In der \gls{bts} finden sich zwei verschiedene Arten von Performancedaten. Da w�ren zum einen die Order-spezifischen, die Daten zu jeder fiktiven Kaufentscheidung in der Vergangenheit generieren und die allgemeinen, die die Performance des Algorithmus �ber den gesamten Zeitraum der Daten widerspiegelt. Wie die einzelnen Daten der beiden Kategorien berechnet werden, soll im Folgenden erl�utert werden.

\subsubsection{Order-spezifische Performancedaten}

Die Order-spezifischen Performancedaten kann man sich auf dem Orders-Tab der \gls{bts} f�r jede Order ansehen. Sie werden wie folgt berechnet:

\begin{itemize}
\item \textbf{Time} \\
	Time gibt den fiktiven Zeitpunkt in der Vergangenheit an, zu dem diese Order unter Echtzeiteinsatz des Algorithmus ausgef�hrt worden w�re.
\item \textbf{Signal Strength} \\
	Gibt die St�rke des Signals (-3 bis +3) aus, das der Algorithmus bei dieser Order gegeben h�tte.
\item \textbf{Position} \\
	Position (Pos) beschreibt die Anzahl an Round Lots die durch die Einstellungen in der \gls{bts} bei der vorliegenden Signal Strength gehandelt werden.
\item \textbf{Price} \\
	Gibt den Preis (P) an zu dem die Order zum gegebenen Zeitpunkt platziert worden w�re. Es handelt sich hierbei um den historischen Close-Wert des entsprechenden Bars. Handelt es sich um eine Kaufentscheidung wird hierzu noch der PricePremiumPercentage addiert, handelt es sich um eine Verkaufsentscheidung wird er subtrahiert.
\item \textbf{Transaction Price} \\
	Der Transaktionspreis (TP) gibt an, wie viel Kapital nun investiert werden muss, um die gegebene Anzahl an Round Lots zum gegebenen Preis (Price) handeln zu k�nnen. Die Round Lots die gekauft werden m�ssen (RLA) sind die Differenz von der alten zur neuen Position. War die alte Position also bspw. -1 und das neue ist 1, dann m�ssen zwei Round Lots gekauft werden. Hierzu werden auch alle zu zahlenden Geb�hren addiert. In der folgenden Gleichung ist RLS die Gr��e eines Round Lots.
	\begin{equation}
		\label{Transaction Price}
		TP = (P * RLS * RLA) + PF
	\end{equation}
\item \textbf{Paid Fee} \\
	Paid Fee (PF) ist die Gesamtheit aus allen relativen (R) und absoluten (A) Transaktionsgeb�hren, die in der \gls{bts} eingestellt wurden f�r diese Order bezahlt h�tten werden m�ssen. Dabei wird klarerweise mit dem TP vor Addition der Geb�hren gerechnet (TP0).
	\begin{equation}
		\label{Paid Fee}
		PF = (\frac{R}{100} * TP0) + A
	\end{equation}
\item \textbf{Gain/Loss [\%]} \\
	GainLoss (GL) beschreibt die prozentuelle Kapitalver�nderung durch vorherige Trades bis zum jetzigen relativ auf das eingesetzte Investitionskapital, also TP. Dabei wird kein unrealisierter Profit/Verlust angeschrieben, sondern nur der wirklich realisierte. Das bedeutet, dass bei einer Verst�rkung der Position der aktuellen Order im Vergleich zu der der letzten GL 0 ist. Wenn dies bei der ersten Order berechnet werden soll, kann man hier nur die Geb�hren durch den TP0 dividieren, da sonst kein realisierter Profit/Verlust zustande kommt:
	\begin{equation}
		\label{Gain/Loss}
		GL = \frac{-PF}{TP0} * 100
	\end{equation}
	Handelt es sich allerdings nicht um die erste Order und die Positionen:
	\begin{equation}
		\label{Gain/Loss}
		GL = \frac{oldPos * (P - oldP - PF)}{oldP} * 100
	\end{equation}
	Wird allerdings bei einem Trade generell nur die Position des letzten Trades verst�rkt, ist GL 0.
\item \textbf{Cumulative Gain/Loss [\%]} \\
	Hierbei werden einfach nur kumulativ die GL-Werte aller bisherigen Orders aufaddiert ausgegeben.
\item \textbf{Portfolio Performance [\%]} \\
	Die Portfolio Performance (PP) ist eine der wichtigsten Performancedaten. Sie gibt an, um wie viel 
\end{itemize}