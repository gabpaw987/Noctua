\chapter{Realisierungsvarianten} \label{chapter:varianten}

\section{Programmiersprachen}
Bei der Programmierung der \gls{bts} ist es wichtig die Interoperabilit�t zu einem dynamisch ladbaren Algorithmus zu erhalten. Weitere wichtige Punkte, die erf�llt werden m�ssen, sind die mathematische Berechnung von performancerelevanten Daten und das effiziente Einlesen und Abarbeiten von Daten. \\
Eine gute Wahl ist die Kombination einer objektorientieren und einer funktionalen Programmiersprache, eine C\#-, F\#-Kombination bietet sich hierbei an. Bei dieser Kombination ist die C\#-Seite die \gls{bts}, welche Daten einliest und Ergebnisse interpretiert. Beim F\#-Softwareprodukt handelt es sich nicht um eine gewohnte Konsolen- oder GUI-Applikation, sondern um eine DLL, welche dynamisch in die \gls{bts} eingebunden werden kann und diese mit Entscheidungen versorgt. Bei dieser L�sung kommt es zu keinem Problem, wenn der Algorithmus (urspr�nglich F\#) auch in einer anderen .Net-Sprache geschrieben wird und als Programmbibliothek exportiert wird. 