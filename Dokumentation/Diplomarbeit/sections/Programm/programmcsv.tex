%!TEX root=../../Benutzerhandbuch.tex
\subsection{Historische Daten}

Um einen Algorithmus mit der \gls{BTS} testen zu k�nnen, m�ssen auch noch historische Aktien-Preisdaten in die Software eingebunden werden, �ber die der Algorithmus getestet wird. Diese k�nnen in Form einer \gls{CSV}-Datei gespeichert und ihr Pfad in der \gls{BTS} angegeben werden. Dabei wurde ein sehr weit verbreitetes Format benutzt, das bspw. auch jegliche Software des renomierten Aktiendatenbereitstellungs-Unternehmens eSignal exportieren kann. Au�erdem werden in dieser Datei die historischen Daten in Form von \glspl{Bar} �ber einen bestimmten Zeitraum (z.B. Daily-\gls{Bar} oder Minute-\gls{Bar}) erwartet und nicht als einzelne Preiswerte. Das Format sieht in etwa so aus:

\begin{lstlisting}[caption=Aufbau der \gls{CSV}-Datei]{csv}
Bar,Date,Time,Open,High,Low,Close
1,01/02/90,00:00,8.8125,9.375, 8.75,9.3125
2,01/03/90,00:00,9.375, 9.50,9.375,9.375
3,01/04/90,00:00,9.375,9.6875,9.3125,9.40625
...
\end{lstlisting}

Zuerst steht also die Nummer des Bars, die allerdings nicht ber�cksichtigt wird. Darauf folgt das Datum in der Form \inline{MM/DD/YY} und die Uhrzeit in der Form \inline{hh:mm}. Zu guter Letzt kommen nun nur noch die Werte Open, High, Low und Close des \glspl{Bar}. Diese Datei kann nahezu unendlich lange gemacht werden, es k�nnen also nahezu unendlich viele Bars nach unten hin erg�nzt werden. Die erste Zeile der Datei wird im Allgemeinen nicht ber�cksichtigt, da sie meist die �berschriften enth�lt. Sollte sich in der ersten Zeile also ein Bar befinden, wird dieser ebenfalls ignoriert.