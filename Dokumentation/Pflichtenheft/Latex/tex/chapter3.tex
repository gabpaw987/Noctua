% Chapter3

\chapter{Produktumgebung} \label{chapter:Produktumgebung}

\section{Software}

Um die \gls{bts} ausf�hren zu k�nnen, wird zu aller erst das installierte .NET-Framework ben�tigt, ohne diesem l�sst sich die Software nicht ausf�hren. Um jedoch auch selber mit dieser Software arbeiten zu k�nnen, ist es ein essentieller Punkt �ber eine Entwicklungsumgebung zu verf�gen, mit der es m�glich F\#-Programme schreiben zu k�nnen. Dieses Programm wird nur auf einem Windows-Rechner lauff�hig und unterst�tzt sein.

\section{Hardware}

Eine Hardwarespezifikation f�r die Software ist nicht notwendig, weil sie auf jedem Windows-Rechner mit installierten .NET-Framework funktionieren kann. Jedoch wird empfohlen einen den momentanen Standards entsprechende Maschine zu benutzen, um etwaige lange Wartezeiten zu verk�rzen und/oder zu beseitigen.

\section{Orgware}

F�r das Durchf�hren einer Analyse eines Algorithmus, ist es notwendig diesen mit spezifischen Daten durchzuf�hren. Der Grund daf�r ist ganz einfach: Man m�chte 2 oder mehrere Algorithmen immer �ber das gleiche historische Fenster laufen lassen, um am Ende 2 mehrere vergleichbare Statistiken hat und den Besseren bestimmen kann.