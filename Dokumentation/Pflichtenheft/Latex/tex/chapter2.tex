% !TeX root = ../Noctua_Pflichtenheft.tex

% Chapter2: Produkteinsatz

\chapter{Produkteinsatz} \label{chapter:produkteinsatz}
\section{Anwendungsbereiche und Zielgruppen}

Der Algorithmus ist prim�r daf�r vorgesehen in eine Softwareumgebung integriert zu werden, die die berechneten Entscheidungen tats�chlich umsetzt. Deshalb sind bei der Entwicklung einerseits die Endanwender, die das fertige Produkt einsetzen, aber auch die Entwickler zu ber�cksichtigen, die diese Integration vornehmen. Die Endanwender sollten ein Mindestwissen �ber Aktienhandel, sowie ein Verst�ndnis der technischen Analyse aufweisen.\\
\\
Als Zielgruppe des Algorithmus sind insbesondere kleine und mittlere Unternehmen (KMU) anvisiert, in denen sich bereits Personen mit finanzwirtschaftlichen Angelegenheiten befassen. Hinzu kommen private Einzelpersonen, die �ber das n�tige Kapital f�r kurzfristigen Aktienhandel verf�gen und mit wenig Aufwand ein Komplettsystem dazu anwenden m�chten.\\
\\
Die \gls{bts} ist zur Unterst�tzung der Systementwicklung vorgesehen und wird folglich von Softwareentwicklern genutzt. Die Benutzbarkeit ist daher nicht das vorrangige Kriterium.

\section{Betriebsbedingungen}

Der Algorithmus wird zur unbeaufsichtigten Entscheidungsberechnung entwickelt, und soll rund um die Uhr laufen k�nnen, soweit dies die Laufzeitumgebung zul�sst. Der weitere Umgang mit den generierten Signalen ist ma�geblich von der anwendenden Software abh�ngig.\\
	Es ist davon auszugehen, dass der Algorithmus im Serverbetrieb eingesetzt wird, weshalb bei der Betrachtung zeitlichen Komponente die Leistung eines modernen Servers als Ausgangspunkt genommen werden kann. Die Entscheidungen m�ssen unter diesem Gesichtspunkt in ausreichender Geschwindigkeit berechnet werden.\\
\\
Die Performance der \gls{bts} h�ngt haupts�chlich von getesteten Algorithmus ab, dar�ber hinaus ist die Geschwindigkeit der Ausf�hrung nicht ausschlaggebend.