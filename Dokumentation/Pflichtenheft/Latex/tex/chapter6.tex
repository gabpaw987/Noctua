% !TeX root = ../Noctua_Pflichtenheft.tex

% Chapter 6: Produktleistungen

\chapter{Produktleistungen} \label{chapter:produktleistungen}
\section{Leistungen des Algorithmus}

/L10/\\
\emph{Minimale Performance}\\
Die Performance des Algorithmus soll statistisch �ber die letzten 3 Jahre einen h�heren Ertrag erzielen, als der risikofreie Zinssatz. Dies soll am Beispiel von 3 �blichen Aktien oder Aktienindizes nachgewiesen werden.\\

% Geschwindigkeit?

\section{Leistungen der Backtesting-Software}

/L20/\\
\emph{Backtesting Kursdaten}\\
In der \gls{bts} sollen mindestens 3 verschiedene Aktienkurse zur Verf�gung stehen, wobei mindestens einer ein Index sein soll.\\
\\
/L30/\\
\emph{Verwendete Daten}\\
Der Algorithmus darf keine zur Zeit noch nicht vorhandenen Informationen zur Entscheidungsfindung verwenden, da dadurch keine Realsituation simuliert wird und keine Informationen zum Echtzeitbetrieb gewonnen werden.

% Was wir alles im Projekt testen m�ssen: sinnvoll als Leistungen zu definieren?
\section{Projektbezogene Leistungen}

Die folgenden Leistungen m�ssen im Laufe des Projektes erf�llt werden und m�ssen sich, abh�ngig von den Ergebnissen, nicht zwangsweise auf das Produkt auswirken.\\
\\
/L40/\\
\emph{Double MA Crossing}\\
Testen der Performance eines Handelssystems mit 2 \glspl{sma}, sowie 2 \glspl{ema}, die durch Kreuzungen Entscheidungen treffen.\\
\\
/L50/\\
\emph{Triple MA Crossing}\\
Testen eines Handelssystems mit 3 beliebigen \glspl{ma}, die durch definierte Stellung zueinander Entscheidungen treffen.\\
\\
/L60/\\
\emph{Support \& Resistance}\\
Testen der Auswirkungen von der Integration von Support- und Resistance-Mechanismen in das Handelssystem.