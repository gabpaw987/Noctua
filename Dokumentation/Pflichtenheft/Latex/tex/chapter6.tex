% !TeX root = ../Noctua_Pflichtenheft.tex

% Chapter 6: Produktleistungen
\chapter{Produktleistungen} \label{chapter:produktleistungen}
\section{Leistungen des Algorithmus}

/L010/ \textbf{Minimale Performance}\\
Die Performance des Algorithmus soll statistisch �ber die letzten 3 Jahre einen h�heren Ertrag erzielen, als der risikofreie Zinssatz. Dies soll am Beispiel von 3 �blichen Aktien oder Aktienindizes nachgewiesen werden.\\

% Geschwindigkeit?

\section{Leistungen der Backtesting-Software}

/L200/ \textbf{Backtesting Kursdaten}\\
In der \gls{bts} sollen mindestens 3 verschiedene Aktienkurse zur Verf�gung stehen, wobei mindestens einer ein Index sein soll.\\
\\
/L030/ \textbf{Verwendete Daten}\\
Der Algorithmus darf keine zur Zeit noch nicht vorhandenen Informationen zur Entscheidungsfindung verwenden, da dadurch keine Realsituation simuliert wird und keine Informationen zum Echtzeitbetrieb gewonnen werden.
