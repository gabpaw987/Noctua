\chapter{Globale Testf�lle}
\label{chapter:Globale_Testf�lle}
/T010/ Ausw�hlen von historischen Daten \\
Es soll m�glich sein, seine eigenen historischen Daten auszuw�hlen und mit diesen den Algorithmus zu testen. Um dies zu bewerkstelligen soll der Benutzer seine eigenen historischen Daten in das Programm ausw�hlen und einf�gen. Diese historischen Daten befinden sich in einer CSV-Datei und werden von dem Programm selbst ausgelesen. \\ \\

/T020/ Ausw�hlen eines eigenen Algorithmus \\
Es soll m�glich sein, seinen eigenen Algorithmus in den Programmablauf einzuf�gen. Damit das funktioniert muss man diesen Algorithmus in F\# programmieren und �ber die GUI einf�gen. Dieser Algorithmus kann nur dann richtig funktionieren, wenn er die Anforderungen des Interfaces adaptiert und damit umgehen kann. \\ \\

/T030/ Erhalten der Ausgaben der \gls{bts} \\
Es soll der gesamte Programmablauf einmal durchgef�hrt werden k�nnen, um das Endprodukt testen zu k�nnen. Der Durchlauf ist nur dann funktionst�chtig und richtig abgelaufen, wenn die Grundfunktion der Software erf�llt ist. Diese sind die Ausgabe der berechneten Performance und des Gewinn/Risiko-Verh�ltnis. Diese Ausgaben sind die essentiellen Messwerte mit denen man dann verschiedene Algorithmen vergleichen kann. \\ \\

/T040/ Fehlermeldung bei fehlerhaften historischen Daten \\
Wenn der \gls{bts} falsche oder fehlerhafte historische Daten in Bezug auf die Datenstruktur, nicht im Hinblick auf deren Inhalt, soll das Programm eine Fehlermeldung ausgeben. Diese soll den Benutzer davor warnen, dass seine Eingabe fehlerhaft ist und mit dieser nicht gearbeitet werden kann. \\ \\

/T050/ Fehlermeldung bei falschen Algorithmus \\
Es soll der \gls{bts} m�glich sein den Algorithmus, der eingef�gt wurde, so weit zu �berpr�fen, dass es erreichbar wird, die F\#-Datei nach der Implementation des Interfaces zu testen. Dieser Test darf nur dann positiv erfolgen, wenn die Datei das gesamte, notwendige Interface implementiert und somit vollst�ndig arbeitsf�hig ist. Falls dies nicht der Fall ist soll das Programm eine Fehlermeldung ausgeben.

